\documentclass[french]{article}
%
% Packages supplementaires utilises
%
\usepackage[T1]{fontenc}
\usepackage{babel}
\usepackage{indentfirst}
\usepackage{graphicx}
\usepackage{epsf}
\usepackage{fancybox}
\usepackage{url}
\usepackage{fancyheadings}
\usepackage{xspace}
%
% Definition des commandes personnalis�es
%[1]{\textsf{#1}\xspace
\newcommand{\obm}{\textsc{obm}\xspace}
\newcommand{\calendar}{\textsc{calendar}\xspace}
\newcommand{\css}{\textsc{css}\xspace}
\newcommand{\fonction}[1]{\textit{#1}\xspace}
\newcommand{\fichier}[1]{\textbf{\textit{#1}}\xspace}
\newcommand{\variable}[1]{\textbf{#1}\xspace}
%
% Modification des marges par defaut
%
\addtolength\textwidth{2cm}
%
% Profondeur de la table des matieres
%
\setcounter{tocdepth}{3}
%
%
%
\pagestyle{fancy}
\lhead[ALIACOM]{ALIACOM}
%
%
%
\title{\Huge
\textbf{OBM : Open Buisness Managment}\\
\vspace{1.5cm}
Manuel d'utilisation\\
\vspace{.5cm}
et\\
\vspace{.5cm}
Documentation technique\\
\vspace{1.5cm}
}
\author{ALIACOM}
%\date{}
%
% Debut du document
%
\begin{document}

\maketitle

\clearpage

\tableofcontents

\clearpage

% Documentation technique d'OBM
% ALIACOM Pierre Baudracco
% $Id$

\part{Manuel technique}
\setcounter{section}{0}

\section{Installation pas � pas d'OBM 1.1.x}

Cette documentation est valide pour la version 1.1.0. Seules les
manipulations sp�cifiques � OBM sont d�crites. Pour l'installation de
PHP, MySQL, PostgreSQL, Apache ou autres composants libres, des exemples sont propos�s correspondant � la version Debian mais il est pr�f�rable de vous r�f�rer � la documentation de votre distribution.


\subsection{Hypoth�ses}

Cette documentation suppose que vous disposez d'un serveur Apache,
d'un serveur de BD MySQL ou PostgreSQL et de PHP correctement configur�s. 

Sous Debian, il faut installer les packages : php4, php4-cgi,
php4-mysql or php4-pgsql, apache, apache-common. 

Dans la suite du document, on suppose que la racine d'OBM se trouve
dans \fichier{/var/www/obm}.

\subsection{T�l�charger les sources}

L'installation serait plus difficile sans...

Il faut donc t�l�charger les sources et les d�compresser dans
\fichier{/var/www} (en fait dans le r�pertoire qui contiendra la
racine de votre OBM).

\shadowbox{
  \begin{minipage}{13cm}
\begin{verbatim}
wget http://obm.aliacom.fr/obm-archives/obm-1.1.0.tar.gz
tar -C /var/www -xzvf obm-1.1.0.tar.gz
cd /var/www
ln -s obm-1.1.0 obm
\end{verbatim}
  \end{minipage}
}

Changer le propri�taire des fichiers sources d'OBM (n�cessaire pour le
safe\_mode activ� dans le module documents). 

\shadowbox{
  \begin{minipage}{13cm}
\begin{verbatim}
cd /var/www/obm
chown -R www-data:www-data *  (sur Debian)
chown -R apache:apache *  (sur Redhat)
\end{verbatim}
  \end{minipage}
}

\subsection{configuration de PHP}

Fichier de configuration de PHP : 

\begin{description}
\item[sur Debian] \fichier{/etc/php4/apache/php.ini}
\item[sur RedHat] \fichier{/etc/php.ini}
\end{description}

La version ligne de commande de PHP doit �galement �tre configur�e
(certains outils d'administration ou certains scripts d'installation
sont aussi utilisables dans ce mode de fonctionnement).

\begin{description}
\item[sur debian] \fichier{/etc/php4/cgi/php.ini}, ou 
\fichier{/etc/php4/cli/php.ini} sur les version r�centes.
\end{description}


La directive include\_path doit �tre renseign�e ainsi (ou contenir au moins ces chemins) : 

\shadowbox{
  \begin{minipage}{13cm}
\begin{verbatim}
include_path = ".:/var/www/obm"
\end{verbatim}
  \end{minipage}
}

Depuis la version 0.7.3, la directive short\_open\_tag peut �tre
positionn�e � Off. Pour les versions ant�rieures, elle doit toutefois
�tre positionn�e � On.

\shadowbox{
  \begin{minipage}{13cm}
\begin{verbatim}
short_open_tag = On (can be switched to Off since OBM 0.7.3)
\end{verbatim}
  \end{minipage}
}

Les variables globales doivent �tre automatiquement enregistr�es : 

\shadowbox{
  \begin{minipage}{13cm}
\begin{verbatim}
register_globals = On
\end{verbatim}
  \end{minipage}
}

La directive Magic\_quote doit �tre � On :

\shadowbox{
  \begin{minipage}{13cm}
\begin{verbatim}
magic_quotes_gpc = On
\end{verbatim}
  \end{minipage}
}

De plus, selon  le propri�taire des fichiers, il faut aussi activer le
safe\_mode (mais OBM peut fonctionner sans �galement) : 

\shadowbox{
  \begin{minipage}{13cm}
\begin{verbatim}
safe_mode = On
\end{verbatim}
  \end{minipage}
}

\subsubsection{V�rification du support Mysql (ou PostgreSQL) par PHP}

Selon le serveur de base de donn�es utilis�, une de ces deux lignes
doit �tre pr�sente dans votre \fichier{php.ini} : 

\shadowbox{
  \begin{minipage}{13cm}
\begin{verbatim}
extension=mysql.so 
extension=pgsql.so
\end{verbatim}
  \end{minipage}
}

Cette modification doit �tre faite pour le module Apache
\emph{et} pour le mode ligne de commande s'ils utilisent des fichiers
\fichier{php.ini} distincts.

\subsection{Configuration d'Apache et de PHP}

Les sources d'OBM sont dans le r�pertoire \fichier{php}, les fichiers
� inclure sont dans le r�pertoire \fichier{obminclude}. Depuis la
version 0.5.2, le r�pertoire \fichier{obminclude} est en dehors de la
racine des documents pour des consid�rations de s�curit�.

Pour l'application web OBM, toutes les configurations Apache et PHP se
font dans le fichier \fichier{httpd.conf} d'Apache.

\subsubsection*{Configurer Apache pour le traitement des fichiers PHP}

Attention, les diff�rents chemins et fichiers peuvent varier entre les distributions et les versions d'apache (apache 1 et 2).

\paragraph{D�finition du type de fichier .php}
�diter le fichier \fichier{httpd.conf} (ou \fichier{apache2.conf}) et
ajouter :

\shadowbox{
  \begin{minipage}{13cm}
\begin{verbatim}
AddType application/x-httpd-php .php
\end{verbatim}
  \end{minipage}
}

\paragraph{Inclure le module php dans Apache}
�diter le fichier \fichier{modules.conf} (ou \fichier{mods-available/php4}) et
ajouter :

\shadowbox{
  \begin{minipage}{13cm}
\begin{verbatim}
LoadModule php4_module /usr/lib/apache2/modules/libphp4.so
\end{verbatim}
  \end{minipage}
}

Sous RedHat 8, avec Apache2, v�rifier que le
\fichier{/etc/httpd/conf.d/php.conf} est correct.

Nous allons  configurer un \emph{virtual host} pour g�rer une instance
d'OBM (vous pouvez avoir plusieurs \emph{virtuals hosts} sur un seul
serveur). 

Dans la section \emph{virtual host} du \fichier{httpd.conf},
positionner le \emph{Document Root} � /var/www/obm/php.

\shadowbox{
  \begin{minipage}{13cm}
\begin{verbatim}
DocumentRoot /var/www/obm/php
\end{verbatim}
  \end{minipage}
}

\subsubsection*{R�pertoires d'include d'OBM}

Le nom du r�pertoire d'include (\fichier{obminclude} par d�faut) est
maintenant une variable d'environnement pour permettre � plusieurs
instances d'OBM bas�es sur le m�me r�pertoire de source de tourner
simultan�ment. Seul ce r�pertoire est � modifier sur chaque instance
car il contient les r�glages sp�cifiques (th�mes, bases de donn�es,
langues) pour chaque instance.

Cette variable est donc positionn�e dans le fichier
\fichier{httpd.conf} d'Apache. Il faut d'abord charger le module
\fichier{env} : 

\shadowbox{
  \begin{minipage}{13cm}
\begin{verbatim}
LoadModule env_module /usr/lib/apache/1.3/mod_env.so (apache 1.3 on Debian)
LoadModule env_module modules/mod_env.so (apache 2 on Redhat)
\end{verbatim}
  \end{minipage}
}

Puis renseigner la variable OBM\_INCLUDE\_VAR avec le nom du
r�pertoire d'include : 

\shadowbox{
  \begin{minipage}{13cm}
\begin{verbatim}
Setenv OBM_INCLUDE_VAR obminclude
\end{verbatim}
  \end{minipage}
}

Le chemin d'acc�s au r�pertoire d'include d'OBM doit �tre donn�
(D'ailleurs, le r�pertoire \fichier{obminclude} peut �tre d�plac�) :  

Renseigner la variable include\_path avec le chemin vers le r�pertoire
d'include : 

\shadowbox{
  \begin{minipage}{13cm}
\begin{verbatim}
php\_value include\_path ".:/var/www/obm"
\end{verbatim}
  \end{minipage}
}


\subsubsection*{Alias pour les images}

Il faut installer un alias images vers le r�pertoire de themes : 

\shadowbox{
  \begin{minipage}{13cm}
\begin{verbatim}
Alias /images /var/www/obm/obminclude/themes
\end{verbatim}
  \end{minipage}
}

\subsubsection*{Directory Index}

Il faut enfin sp�cifier le fichier par d�faut : 

\shadowbox{
  \begin{minipage}{13cm}
\begin{verbatim}
DirectoryIndex obm.php
\end{verbatim}
  \end{minipage}
}

Nous recommandons \textbf{chaudement} d'interdire l'acc�s direct aux fichiers
\fichier{.inc} :

\shadowbox{
  \begin{minipage}{13cm}
\begin{verbatim}
<Files ~ "\.inc$">
   Order allow,deny
   Deny from all
</Files>
\end{verbatim}
  \end{minipage}
}


\subsubsection*{Section virtual host d'exemple compl�te}

V�rifier que votre IP soit d�finie comme un \emph{named virtual
  host} et ins�rer cette section :

\shadowbox{
  \begin{minipage}{13cm}
\begin{verbatim}
NameVirtualHost 192.168.1.5

<VirtualHost 192.168.1.5>
 ServerAdmin root@localhost
 DocumentRoot /var/www/obm/php
 ServerName obm
 ErrorLog /var/log/apache/obm-error.log
 CustomLog /var/log/apache/obm-access.log common
 SetEnv OBM_INCLUDE_VAR obminclude
 Alias /images /var/www/obm/obminclude/themes
 DirectoryIndex obm.php
</VirtualHost>
\end{verbatim}
  \end{minipage}
}


\subsection{Configuration d'\obm}

La configuration d'\obm se trouve dans le fichier \fichier{obminclude/obm\_conf.inc}.

Pour le cr�er, copier le fichier \fichier{obminclude/obm\_conf.inc.sample} dans le nouveau fichier \fichier{obminclude/obm\_conf.inc}, et �diter ce dernier.


\subsubsection{Configuration initiale pour la base de donn�es OBM}

�diter le fichier \fichier{obm\_conf.inc} dans \fichier{obminclude} et

\begin{itemize}
\item choisir la base de donn�es � utiliser ; 
\item la d�clarer ;
\item d�clarer le nom d'utilisateur et le mot de passe � utiliser.
\item d�clarer l'url d'acc�s � OBM (\$cgp\_host).
\item d�clarer le d�p�t de documents (voir \ref{install_doc}).
\end{itemize}

Il est �galement possible de modifier des pr�f�rences globales dans ce
fichier, comme cgp\_mail\_enabled par exemple.

\shadowbox{
  \begin{minipage}{13cm}
\begin{verbatim}
// Database infos
$obmdb_host = "localhost";
$obmdb_dbtype = "PGSQL"; // (MYSQL || PGSQL)
$obmdb_db = "obm";
$obmdb_user = "obm";
$obmdb_password = "obm";

$cgp_host = "http://obm/";
...
$cdocument_root = "/var/www/obmdocuments/";
...
// is Mail enabled ? (agenda)
$cgp_mail_enabled = false;
\end{verbatim}
  \end{minipage}
}


\subsubsection{Configurer le d�p�t de documents}
\label{install_doc}

�diter le fichier \fichier{obm\_conf.inc} dans \fichier{obminclude}.

\shadowbox{
  \begin{minipage}{13cm}
\begin{verbatim}
$cdocument_root = "/var/www/obmdocuments";
\end{verbatim}
  \end{minipage}
}


Cr�er le r�pertoire et le faire appartenir � l'utilisateur ex�cutant le serveur apache (www-data sous Debian).

\shadowbox{
  \begin{minipage}{13cm}
\begin{verbatim}
mkdir /var/www/obmdocuments
chown www-data:www-data /var/www/obmdocuments
\end{verbatim}
  \end{minipage}
}


\subsubsection{Configuration et cr�ation de la base de donn�es}

Les scripts de cr�ation de la base de donn�es se trouvent dans le
r�pertoire \fichier{scripts/1.1}.

\subsubsection*{Langue par d�faut des donn�es}

Certaines valeurs de r�f�rences sont ins�r�es par d�faut. Ces valeurs
peuvent �tre en fran�ais (par d�faut) ou en anglais. Pour passer de
l'un � l'autre, il faut �diter le fichier
\fichier{scripts/1.1/install\_obmdb\_1.1.mysql.sh} (ou
\fichier{scripts/1.1/install\_obmdb\_1.1.pgsql.sh}) pour changer la
valeur de DATA\_LANG � \texttt{en} : 

\shadowbox{
  \begin{minipage}{13cm}
\begin{verbatim}
# Mysql User, Password and Data lang var definition 
U=obm 
P="obm" 
DB="obm" 
DATA_LANG="en"
\end{verbatim}
  \end{minipage}
}

\subsubsection{MySQL}

Pr�-requis : disposer d'un login/mot de passe valide pour MySQL.

Exemple de cr�ation d'un utilisation MySQL : 

\shadowbox{
  \begin{minipage}{13cm}
\begin{verbatim}
GRANT CREATE, DROP, SELECT, UPDATE, INSERT, DELETE, LOCK TABLES, INDEX
ON obm.* TO obm@localhost IDENTIFIED BY 'obm'; 
\end{verbatim}
  \end{minipage}
}
ou plus simplement :

\shadowbox{
  \begin{minipage}{13cm}
\begin{verbatim}
GRANT ALL ON obm.* TO obm@localhost IDENTIFIED BY 'obm';
\end{verbatim}
  \end{minipage}
}

Le script \fichier{install\_obmdb\_1.1.mysql.sh} g�re les diff�rentes
�tapes de cr�ation de la base de donn�es. Positionner les bonnes
valeurs de login/mot de passe dans ce script et le lancer :

\shadowbox{
  \begin{minipage}{13cm}
\begin{verbatim}
cd obm/scripts/1.1
emacs install_obmdb_1.1.mysql.sh
./install_obmdb_1.1.mysql.sh
\end{verbatim}
  \end{minipage}
}

\subsubsection{PostgreSQL}

OBM supporte PostgreSQL depuis la version 0.7.5.

Cr�ation de l'utilisateur obm et de la base de donn�es �ponyme :

\shadowbox{
  \begin{minipage}{13cm}
\begin{verbatim}
create user obm password 'obm';
create database obm with owner = obm;
\end{verbatim}
  \end{minipage}
}

\subsubsection*{Jeux de caract�res}

Si votre server PostgreSQL n'est pas configur� pour le jeu de
caract�res latin1, �ditez le fichier \fichier{scripts/1.1/postgres-pre.sql}
et d�commentez la ligne :

\shadowbox{
  \begin{minipage}{13cm}
\begin{verbatim}
\encoding latin1
\end{verbatim}
  \end{minipage}
}

\subsubsection*{Cr�ation de la base Postgres}

Le script \fichier{install\_obmdb\_1.1.pgsql.sh} g�re toutes les
�tapes de cr�ation de la base de donn�es.

Ajoutez les informations login/mot de passe (et base de donn�es) dans
ce script et lancez-le :

\shadowbox{
  \begin{minipage}{13cm}
\begin{verbatim}
cd obm/scripts/1.1
emacs install_obmdb_1.1_pgsql.sh
./install_obmdb_1.1_pgsql.sh
\end{verbatim}
  \end{minipage}
}

\subsection{Support du syst�me d'authentification CAS (Central Authentication Service)}

\subsubsection{Pr�-requis}

Le support CAS s'appuie sur la librairie cliente PHP, phpCAS (http://esup-phpcas.sourceforge.net/index.html).

Les librairies Curl doivent �tre install�es sur votre syst�me et compil�es avec le support de SSL.
Sous Debian, il faut installer les packages : curl, libcurl3, libcurl3-gnutls.  

La version de PHP doit �tre sup�rieure � la 4.2.2.
PHP doit �tre install� avec le support curl, openssl, domxml et zlib.
Sous Debian, il faut installer les packages php4-curl et php4-domxml.

Il est n�cessaire d'installer PEAR. Les libariries PEAR::DB et PEAR:Log sont utilis�es.
Sous Debian, il faut installer le packages php4-pear.

\subsubsection{Configuration d'OBM}

�diter le fichier \fichier{obm\_conf.inc} dans \fichier{obminclude} et

\begin{itemize}
\item sp�cifier le mode d'authentification "CAS"; 
\item d�clarer les param�tres de connection au serveur CAS.
\end{itemize}

Les variables suivantes du fichier de configuration doivent �tre rennseign�es:

\shadowbox{
  \begin{minipage}{13cm}
\begin{verbatim}

...
// authentification : 'CAS' (SSO) or 'standalone' (default)
$auth_kind="CAS";
$cas_server = "sso.aliacom.local";
$cas_server_port = "8443";
$cas_server_uri = "/cas";
...

\end{verbatim}
  \end{minipage}
}

\subsection{Lancer OBM (acc�der � obm.php depuis un navigateur)}

D'abord, red�marrer le serveur web apache : 

\shadowbox{
  \begin{minipage}{13cm}
\begin{verbatim}
/etc/init.d/apache restart
\end{verbatim}
  \end{minipage}
}

Si tout marche bien, lancer un navigateur (Firefox par exemple) et
aller � l'URL : \url{http://yourvirtualhost/}, puis se connecter en
utilisant le compte \texttt{uadmin/padmin}.


% Documentation technique d'OBM : Configuration specifique � un site
% ALIACOM Pierre Baudracco
% $Id$

\section{Configuration sp�cifique d'un site}

Depuis la version 0.8, OBM est pens� pour faciliter la personnalisation d'une mise en oeuvre en restant compatible avec le produit g�n�rique.

Pour ce faire les options de configurations et les donn�es sp�cifiques d'un site sont situ�es dans un fichier et un r�pertoire d�di�s.\\

\begin{tabular}{|p{7cm}|p{5cm}|}
\hline
Fichier de configuration sp�cifique & obminclude/obm\_conf.inc \\
\hline
R�pertoire de configuration sp�cifique & obminclude/site \\
\hline
\end{tabular}

\subsection{Le param�trage des sections visibles}

Le tableau \$cgp\_show[``section''] d�finit pour chaque section si celle-ci doit �tre affich�e ou non.
Par d�faut toutes les sections sont affich�es.

Pour ne pas afficher une section il faut ins�rer dans le fichier de configuration la ligne :

\begin{verbatim}  
$cgp_show[``section''][``com''] = false;
\end{verbatim}  
\vspace{0.3cm}

Liste des sections disponibles :\\
\begin{itemize}
\item com
\item prod
\item compta
\item user
\item admin
\item aliamin
\end{itemize}


\subsection{Le param�trage des modules visibles}

Le tableau \$cgp\_show[``module''] d�finit pour chaque module si celui-ci doit �tre affich� ou non.
Par d�faut tous les modules sont affich�s.

Pour ne pas afficher un module il faut ins�rer dans le fichier de configuration la ligne :

\begin{verbatim}  
$cgp_show[``module''][``deal''] = false;
\end{verbatim}  
\vspace{0.3cm}

Liste des modules disponibles :\\
\begin{itemize}
\item company
\item contact
\item deal
\item list
\item agenda
\item todo
\item publication
\item time
\item project
\item contract
\item incident
\item document
\item account
\item invoice
\item payment
\item settings
\item user
\item group
\end{itemize}


\subsection{Le param�trage des champs visibles}

Afin de permettre des utilisations d'\obm d'orientations diverses, certains modules peuvent �tre param�tr�s afin de ne pas afficher et g�rer tous les champs pr�vus.

A partir de la version 0.8 d'\obm, les modules \company et \contact ont des champs param�trables.\\

Le tableau \$cgp\_hide[``module''] d�finit pour les modules cibles les champs devant �tre cach�s.
Par d�faut tous les champs sont affich�s et il faut donc pr�ciser explicitement les champs non souhait�s.

Pour ne pas afficher un champ il faut ins�rer dans le fichier de configuration la ligne :

\begin{verbatim}  
$cgp_hide[``company][``number] = true;
\end{verbatim}  

Lorsqu'un champ est cach�, il n'est plus disponible dans les crit�res de recherche, dans les listes de r�sultats ni dans les fiches de d�tail.


\subsubsection{Les champs param�trables du module \company}

Liste des champs param�trables (masquables) du module \company :

\begin{verbatim}
// Company module
$cgp_hide["company"]["number"] = true;                 // num�ro
$cgp_hide["company"]["companytype_label"] = true;      // type
$cgp_hide["company"]["companyactivity_label"] = true;  // secteur d'activit�
$cgp_hide["company"]["category"] = true;               // cat�gorie
$cgp_hide["company"]["address3"] = true;               // ligne adresse 3
\end{verbatim}  


\subsubsection{Les champs param�trables du module \contact}

Liste des champs param�trables (masquables) du module \contact :

\begin{verbatim}
// Contact module
$cgp_hide["contact"]["contact_function"] = true;  // fonction
$cgp_hide["contact"]["contact_title"] = true;     // titre
$cgp_hide["contact"]["category1"] = true;         // cat�gorie 1
$cgp_hide["contact"]["category2"] = true;         // cat�gorie 2
$cgp_hide["contact"]["address3"] = true;          // ligne adresse 3
\end{verbatim}  


% Documentation technique d'OBM : utilisation des css
% ALIACOM Medhi Rande
% $Id 

\section{Les feuilles de style - CSS}
\label{display}

Le code produit par OBM est et doit �tre conforme XHTML 1.0 Transitionnal.
Les feuilles de style (CSS) sont utilis�es selon le standard CSS2. (1 ?).\\

OBM d�fini des zones graphiques (bandeau de gauche, bandeau de droite,...) ainsi que des objects graphiques (block des sections, fiche de d�tail, block des derni�res entit�s visit�es,...). Une fonction g�n�rale d'affichage \textbf{display\_page()}, sp�cifique � chaque th�me indique la r�partition des objets dans les zones.

\subsection{Les zones graphiques d�finies dans le th�me standard}

L'affichage d'OBM est effectu� dans des zones d�finies.
Le th�me standard d�fini les zones :\\

\begin{tabular}{|c|l|}
\hline
\textbf{Zone} & \textbf{Description pour th�me standard}\\
\hline
header & Bandeau du haut \\
\hline
leftpanel & Bandeau de gauche \\
\hline
detailpanel & Zone principale d'affichage, soit unique soit d�coup�e en middlepanel et rightpanel\\
\hline
middlepanel & Zone centrale d'affichage quand un bandeau est pr�sent � droite (rightpanel)\\
\hline
rightpanel & Bandeau de droite. Si non vide, alors affich� ainsi que middle panel\\
\hline
\end{tabular}

\subsection{Les objets graphiques d�finis}

L'affichage d'OBM est d�coup� selon les objets graphiques suivants, pour lesquels des styles sont d�finis :\\

\begin{tabular}{|c|c|p{8.8cm}|}
\hline
\textbf{Objet} & \textbf{Style} & \textbf{Description}\\
\hline
en-t�te HTML & head & En-t�te de d�finition du type de document, titre (<head>)\\
\hline
en-t�te & header & Block d'en-t�te contenant le logo, les infos de connexions\\
\hline
section & section & Block affichant les sections (onglets par exemple)\\
\hline
module & module & Block contenant les modules d'une section\\
\hline
action & action & Block contenant les actions disponibles du module\\
\hline
\multirow{4}{3cm}{bloc g�n�rique} & block & Bloc g�n�rique\\
\cline{2-3}
 & blockTitle & Titre du bloc \\
\cline{2-3}
 & blockItem & �l�ments du bloc \\
\hline
\multirow{4}{3cm}{derni�re visite} & last & Bloc contenant les entit�s derni�rement visit�es\\
\cline{2-3}
 & lastTitle & Titre du block \\
\cline{2-3}
 & lastItem & �l�ments de la liste des entit�s visit�es \\
\hline
titre & title & Block contenant le titre de la page (ex: Soci�t� : Chercher)\\
\hline
message & msg & Block contenant la zone de message\\
\hline
\multirow{3}{3cm}{recherche} & search & Block contenant le formulaire de recherche\\
\cline{2-3}
 & searchForm & Donn�e d'un champ de recherche \\
\cline{2-3}
 & searchLabel & Label d'un champ de recherche \\
\hline
\multirow{5}{3cm}{d�tail} & detail & Block contenant le d�tail d'une fiche d'un module\\
\cline{2-3}
 & detailHead & En-t�te d'une fiche d�tail \\
\cline{2-3}
 & detailLabel & Label d'une fiche d�tail \\
\cline{2-3}
 & detailText & Donn�e d'une fiche d�tail \\
\cline{2-3}
 & detailForm & Donn�e d'un formulaire d'une fiche d�tail en modification \\
\hline
infos cr�ation & detailInfo & Utilisateurs et dates de cr�ation / modification\\
\hline
\multirow{2}{3cm}{boutons d�tail} & detailButton & Block contenant les boutons associ�s � une fiche d�tail\\
\cline{2-3}
 & detailButtons & Bouton d'une fiche d�tail \\
\hline
\multirow{3}{3cm}{link} & link & Block contenant les liens vers les entit�s associ�es\\
\cline{2-3}
 & linkTitle & Titre du block \\
\cline{2-3}
 & linkList & �l�ments de la liste des entit�s associ�es \\
\hline
\multirow{8}{3cm}{r�sultat de recherche} & result & Block contenant un r�sultat de recherche\\
\cline{2-3}
 & resultIndex & Block contenant l'index d'un r�sultat de recherche \\
\cline{2-3}
 & resultIndexIcon & ic�nes de l'index d'un r�sultat de recherche \\
\cline{2-3}
 & resultHead & Titre des colonnes de r�sultat \\
\cline{2-3}
 & resultText & Cellules du tableau de r�sultats \\
\cline{2-3}
 & resultTextC & Cellules centr�es du tableau de r�sultats \\
\cline{2-3}
 & resultPref & table des pr�f�rences d'affichage d'un r�sultat de recherche \\
\cline{2-3}
 & resultPrefHide & Cellules des champs d�sactiv�s des pr�f�rences d'affichage \\
\hline
\multirow{4}{3cm}{formulaire d'admin} & admin & Block contenant un formulaire d'administration\\
\cline{2-3}
 & adminHead & titre du formulaire d'administration \\
\cline{2-3}
 & adminLabel & label d'un formulaire d'administration \\
\cline{2-3}
 & adminText & Donn�e d'un formulaire d'administration \\
\hline
fin HTML & end & Cloture de la page\\
\hline
\end{tabular}


\clearpage
\section{Sp�cifications globales}
% Documentation technique d'OBM : Gestion des entit�s priv�es
% ALIACOM Pierre Baudracco
% $Id$


%%\clearpage
\subsection{Gestion des entit�s priv�es et visibilit�}

La gestion des entit�s priv�es est standardis�e afin de permettre une �volution possible des r�gles de visibilit� des entit�s.


\subsubsection{Sp�cifications de la notion de visibilit�}

Une api tr�s simple (1 fonction d�finie dans global.inc) est disponible :\\

\shadowbox{
\begin{minipage}{13cm}
\begin{verbatim}
is_entity_visible($entity, $obm_q, $uid='''')
\end{verbatim}
\end{minipage}
}
\begin{itemize}
 \item \textbf{\$entity} : entit� � v�rifier.\\
Exemple : ``company'', ``contact''... Attention ce n'est pas toujours le module (ex: parendeal dans module deal)
 \item \textbf{\$obm\_q} : Objet base de donn�es contenant l'entit�\\
Exemple : souvent l'objet associ� � la requete : run\_query\_detail()
 \item \textbf{\$uid} (optionnel) : uid de l'utilisateur pour lequel le droit de visibilit� doit �tre v�rifi�.
Si non donn�, l'utilisateur courant est utilis�.
\end{itemize}

\vspace{0.4cm}

Actuellement une entit� est visible par un utilisateur si l'entit� est publique (visibility == 0) ou que l'utilisateur en est le cr�ateur.

Impl�mentation du test de visibilit� :
\begin{verbatim}
  $field_vis = "${entity}_visibility";
  $field_uc = "${entity}_usercreate";

  if ( ($q->f("$field_vis") == 0)
    || ($q->f("$field_uc") == $uid) ) {
    return true;
  } else {
    return false;
  }
\end{verbatim}


\subsubsection{Modules g�rant la notion de visibilit�}

\begin{tabular}{|p{5cm}|c|}
\hline
\textbf{Module} & \textbf{Depuis version OBM} \\
\hline
contact & 0.4.0 \\
\hline
deal & 0.4.0 \\
\hline
list & 0.8.2 \\
\hline
\end{tabular}
% Documentation technique d'OBM : Appels de modules externes (ext_)
% ALIACOM Pierre Baudracco
% $Id$


\subsection{Appels de modules externes, popups}
\label{extmod}

L'appel de modules externes permet depuis un module d'interroger ou d'obtenir des informations d'un autre module.
Exemple : pour ajouter un utilisateur � un groupe, le module groupe fait appel au module utilisateur.\\

Afin de permettre une impl�mentation unique du module appel� qui sera utilis�e par tous les modules le n�cessitant, l'appel de modules externes est standardis� et r�pond aux objectifs suivants :\\
\begin{itemize}
\item Pas de dupplication de code inutile et lourde � maintenir
\item Le module appel� propose une interface ind�pendante du module appelant
\item Diff�rents types d'informations peuvent �tre demand�es (un Id, une liste d'Id)
\item Le module appelant doit r�cup�rer les informations (par appel � une url ou dans des widgets)
\end{itemize}



\subsubsection{Description des diff�rents appels externes}

Un appel externe est caract�ris� par une action du module appel�.
3 types d'appels sont actuellement d�finis.
Pour chaque appel, divers param�tres sont � fournir.\\

\begin{tabular}{|p{3cm}|p{9.5cm}|}
\hline
\textbf{Action} & \textbf{Commentaire} \\
\hline
ext\_get\_ids & Affiche sans bandeau, en popup, la recherche des entit�s du module, permet d'effectuer des recherches et de renvoyer les entit�s s�lectionn�es par des cases � cocher � une url \\
\hline
ext\_get\_id & Affiche sans bandeau, en popup, la recherche des entit�s du module, permet d'effectuer des recherches et de renvoyer l'entit� s�lectionn�e en cliquant dessus � une url ou dans un widget\\
\hline
ext\_get\_id\_url & Voir difference reelle avec ext\_get\_id ? uniquement param ext\_url ? \\
\hline
\end{tabular}

\subsubsection{Sp�cifications de l'action ext\_get\_ids}

L'exemple donn� est l'ajout d'utilisateurs � un groupe.
Le module appelant est \group, le module appel� est \user.
Depuis un groupe, on s�lection l'ajout d'utilisateurs. Une fen�tre externe popup de recherche d'utilisateurs s'ouvre.\\

L'action ``Ajout d'utilisateurs'' doit donc �tre d�finie dans le module groupe comme un appel externe au module utilisateur.
Des param�tres doivent �tre pass�s.\\

\begin{tabular}{|p{2.5cm}|p{6cm}|p{4.5cm}|}
\hline
\textbf{Param�tre} & \textbf{Commentaire} & \textbf{Exemple} \\
\hline
\multicolumn{3}{|c|}{\textbf{Param�tres de configuration du module appel�}}\\
\hline
action & ext\_get\_ids & \\
\hline
popup & (0 | 1) affichage en popup (sans menu) & 1 \\
\hline
ext\_title & Titre affich� dans la fen�tre externe & Ajouter des utilisateurs\\
\hline
\multicolumn{3}{|c|}{\textbf{Param�tres : retour par url + action}}\\
\hline
ext\_action & Action appel�e par la fen�tre externe & user\_add\\
\hline
ext\_target & Fen�tre cible de retour (target) & Groupe \\
\hline
ext\_url & Url appel�e par la fen�tre externe & \$path/group/group\_index.php \\
\hline
ext\_id & Id de l'entit� r�ceptionnant la r�ponse & Id du groupe origine \\
\hline
\multicolumn{3}{|c|}{\textbf{Retour par widget (select) : fonction javascript fill\_ext\_form() }}\\
\hline
ext\_widget & Indique le widget (select) devant s�lectionner les donn�es & \\
\hline
\multicolumn{3}{|c|}{\textbf{Retour par insertion d'�l�ments (div) : fonction javascript fill\_ext\_element() }}\\
\hline
ext\_element & Indique l'�l�ment (widget) devant s�lectionner les donn�es & \\
\hline
\end{tabular}


\subsubsection{Sp�cifications de l'action ext\_get\_id}

L'exemple donn� est la s�lection d'une soci�t� lors de la cr�ation d'une affaire ou contact.
Le module appelant est \deal, le module appel� est \company.
Depuis le module \deal, on s�lectionne la cr�ation d'une nouvelle affaire. Une fen�tre externe popup de recherche de soci�t� s'ouvre.\\

L'action ``Nouvelle'' doit donc �tre d�finie dans le module \deal comme un appel externe au module \company.
Des param�tres doivent �tre pass�s.\\

\begin{tabular}{|p{2.5cm}|p{6cm}|p{4.5cm}|}
\hline
\textbf{Param�tre} & \textbf{Commentaire} & \textbf{Exemple} \\
\hline
\multicolumn{3}{|c|}{\textbf{Param�tres de configuration du module appel�}}\\
\hline
action & ext\_get\_id & \\
\hline
popup & (0 | 1) affichage en popup (sans menu) & 1 \\
\hline
ext\_title & Titre affich� dans la fen�tre externe & S�lectionner une soci�t�\\
\hline
\multicolumn{3}{|c|}{\textbf{Retour par url : fonction javascript check\_get\_id\_url()}}\\
\hline
ext\_url & Url appel�e par la fen�tre externe (l'url n'est pas ferm�e car elle recevra l'id de l'entit� s�lectionn�e) & \$path/deal/deal\_index.php? action=new\&amp; param\_company= \\
\hline
\multicolumn{3}{|c|}{\textbf{Retour par widget : fonction javascript check\_get\_id()}}\\
\hline
ext\_widget & Indique le widget (hidden) devant recevoir l'Id & f\_contact.company\_new\_id \\
\hline
ext\_widget\_text & Indique le widget (textfield) devant recevoir le label ou nom correspondant � l'Id s�lectionn� & f\_contact.company\_new\_name \\
\hline
\end{tabular}


\subsubsection{D�finition d'un appel � un module externe}

Ce peut �tre une action d'un menu (ex: ajouter utilisateur � groupe) ou un lien direct (ajouter une cat�gorie � une soci�t� depuis le formulaire de mise � jour soci�t�).

\paragraph{Appel par d�finition d'une action}
Exemple : Ajouter un utilisateur au groupe courant.\\

\shadowbox{
\begin{minipage}{14cm}
\begin{verbatim}
// Sel user add : Users selection
  $actions["GROUP"]["sel_user_add"] = array (
    'Name'     => $l_header_add_user,
    'Url'      => "$path/user/user_index.php?action=ext_get_ids&amp;popup=1
&amp;ext_title=".urlencode($l_add_user)."&amp;ext_action=user_add
&amp;ext_url=".urlencode($path."/group/group_index.php")."
&amp;ext_id=".$group["id"]."&amp;ext_target=$l_group",
    'Right'    => $cright_write,
    'Popup'    => 1,
    'Target'   => $l_group,
    'Condition'=> array ('detailconsult','user_add','user_del','group_add'...) 
                                    	  )
\end{verbatim}
\end{minipage}
}
\begin{itemize}
 \item \textbf{Popup} : indique si l'action doit ouvrir une fen�tre externe (1) ou se d�rouler dans la fen�tre courante (d�faut ou 0).\\
 \item \textbf{Target} : uniquement si Popup est positionn�, indique le nom donn� � la fen�tre source avant ouverture du Popup. N�cessaire dans le cas d'utilisation d'ext\_target pour identifier la fen�tre retour. Ces deux valeurs doivent �tre identiques.\\
\end{itemize}

\paragraph{Appel par lien direct}
Exemple : Ajouter des groupes au rendez-vous courant.\\

\shadowbox{
\begin{minipage}{14cm}
\begin{verbatim}
  $url2 = "$path/group/group_index.php?action=ext_get_ids&amp;popup=1
&amp;ext_widget=forms[0].elements[6]
&amp;ext_title=" . urlencode($l_agenda_select_group);
  ...
  $block = ``
     <a href=\"javascript: return false;\" onclick=\"window.open('$url2',
'','height=$popup_height,width=$popup_width,scrollbars=yes'); return false;\">
<img src=\"/images/$set_theme/$ico_add_group\" /></a></td>
  ...``;
\end{verbatim}
\end{minipage}
}


\subsubsection{Modules impl�mentant des appels externes}

\begin{tabular}{|p{5cm}|c|c|}
\hline
\textbf{Module} & \textbf{Appel} & \textbf{Depuis version OBM} \\
\hline
\user & ext\_get\_ids & 0.7\\
\hline
\group & ext\_get\_ids & 0.8 \\
\hline
\resource & ext\_get\_ids & 1.0\\
\hline
\company & ext\_get\_id & 0.7 \\
\hline
\contact & ext\_get\_ids & 0.7 \\
\hline
\deal & ext\_get\_id & 0.8.6 \\
\hline
\List & ext\_get\_id & 0.8 \\
\hline
\publication & ext\_get\_id & 0.8 \\
\hline
\project & ext\_get\_id & 0.8 \\
\hline
\contract & ext\_get\_id & 0.8 \\
\hline
\doc & ext\_get\_ids & 0.8 \\
\hline
\end{tabular}


\subsubsection{Impl�mentation dans un module}

\paragraph{D�finition des actions}
Une action doit �tre d�clar�e pour �tre autoris�e.
Il faut donc d�clarer les actions externes.
Il n'y a pas besoin d'indiquer de titre, l'action ne disposant pas de menu (appel depuis un module externe).\\

\shadowbox{
\begin{minipage}{14cm}
\begin{verbatim}
// Ext get Ids : External User selection
  $actions["USER"]["ext_get_ids"] = array (
    'Right'    => $cright_read,
    'Condition'=> array ('none')
                                    );
\end{verbatim}
\end{minipage}
}



\paragraph{Affichage all�g� (menu, liens)}
L'affichage des fen�tres externes, en popup, poss�de quelques particularit�s afin d'am�liorer l'ergonomie d'utilisation :\\
\begin{itemize}
\item Pas de bandeaux ou menus
\item Pas de liens, donn�es suppl�mentaires (formulaire de s�lection,...)
\end{itemize}
\vspace{0.3cm}

Le bandeau ou menu g�n�ral ne sera affich� que si le param�tre popup n'est pas positionn� (module\_index.php).\\

\shadowbox{
\begin{minipage}{14cm}
\begin{verbatim}
if (! $obm_user["popup"]) {
  $display["header"] = display_menu($module);
}
\end{verbatim}
\end{minipage}
}
\vspace{0.3cm}

Les fonctions d'affichage utilis�es par les appels externes sont les m�mes que les fonctions d'affichage classiques du module (ex: affichage de la liste des utilisateurs apr�s une recherche).\\

Il faut donc que ces fonctions tiennent compte de ces contraintes. L'affichage sans liens est support� par la classe d'affichage OBM\_DISPLAY.

De m�me, le code de g�n�ration du formulaire de r�cup�ration des donn�es (ex: s�lection des utilisateurs) est � cr�er quand n�cessaire.\\

\shadowbox{
\begin{minipage}{14cm}
\begin{verbatim}
  $user_d = new OBM_DISPLAY("DATA", $pref_q, "user");
  if ($popup) {
    $user_d->display_link = false;
    $user_d->data_cb_text = "X";
    $user_d->data_idfield = "userobm_id";
    $user_d->data_cb_name = "data-u-";
    if ($ext_widget != "") {
      $user_d->data_form_head = "
      <form onsubmit=\"fill_ext_form(this); return false;\">";
    } else {
      $user_d->data_form_head = "
      <form target=\"$ext_target\" method=\"post\" action=\"$ext_url\">";
    }
    $user_d->data_form_end = "
      <p class=\"detailButton\">
        <p class=\"detailButtons\">
        <input type=\"submit\" value=\"$l_add\" />
        <input type=\"hidden\" name=\"ext_id\" value=\"$ext_id\" />
        <input type=\"hidden\" name=\"action\" value=\"$ext_action\" />
        </p>
      </p>
      </form>'';

    $display_popup_end = "
      <p>
      <a href=\"\" onclick='window.close();'>$l_close</a>
      </p>";
  }
\end{verbatim}
\end{minipage}
}


\paragraph{Transport des param�tres}

Lorsque une navigation est possible dans une fen�tre externe (ex: cas de recherche puis liste de r�sultat) les param�tres externes doivent �tre transmis afin d'�tre conserv�s.\\

\shadowbox{
\begin{minipage}{14cm}
\begin{verbatim}
  if ($popup) {
    $ext_action = $user["ext_action"];
    $ext_target = $user["ext_target"];
    $ext_url = $user["ext_url"];
    $ext_id = $user["ext_id"];
    $url_ext = "&amp;ext_action=$ext_action&amp;ext_url=$ext_url
&amp;ext_id=$ext_id&amp;ext_target=$ext_target";
  }

  $url = url_prepare("user_index.php?action=search&amp;tf_login=$login
&amp;tf_lastname=$lname&amp;sel_perms=$perms&amp;cb_archive=$archive$url_ext");
\end{verbatim}
\end{minipage}
}

\paragraph{Gestion du retour javascript}

Une fois la ou les entit�s s�lectionn�es dans la fen�tre externe, l'appel retour est r�alis� soit par l'url et l'action indiqu�es, soit � l'aide de fonctions javascript, telles que r�f�renc�es dans les tableaux de sp�cifications des actions ext\_get\_ids et ext\_get\_id.


\paragraph{action ext\_get\_ids, retour par widget}
Cette fonction est d�finie uniquement si le param�tre \$ext\_widget est rempli afin d'�viter des erreurs javascript.

\shadowbox{
\begin{minipage}{14cm}
\begin{verbatim}
if ($ext_widget != "") {
  $extra_js .= "

function fill_ext_form(int_form) {
   size = int_form.length;
   ext_field = window.opener.document.$ext_widget;
   for(i=0; i <size ; i++) {
     if(int_form.elements[i].type == 'checkbox'){
       if(int_form.elements[i].checked == true) {
	 ext_size = ext_field.length;
	 for(j=0; j< ext_size; j++) {
	   if('cb_g' + ext_field.options[j].value == int_form.elements[i].name) {
	     window.opener.document.$ext_widget.options[j].selected =true;
	   }
	 }
       }
     }
   }
}";
}
\end{verbatim}
\end{minipage}
}


\paragraph{action ext\_get\_id, retour par widget}
Cette fonction est d�finie uniquement si les param�tres \$ext\_widget sont remplis afin d'�viter des erreurs javascript.

\shadowbox{
\begin{minipage}{14cm}
\begin{verbatim}
if (($ext_widget != "") || ($ext_widget_text != "")) {
  $extra_js .= "

function check_get_id(valeur,text) {
  if ((valeur < 1) || (valeur == null)) {
    alert (\"$l_j_select_company\");
    return false;
  } else {
    window.opener.document.$ext_widget.value=valeur;
    window.opener.document.$ext_widget_text.value=text;
    window.close();
    return true;
  }
}";
}
\end{verbatim}
\end{minipage}
}


\paragraph{action ext\_get\_id, retour par url}
L'id de l'entit� s�lectionn�e est ajout�e � l'url de retour.

\shadowbox{
\begin{minipage}{14cm}
\begin{verbatim}
function check_get_id_url(p_url, valeur) {
  if ((valeur < 1) || (valeur == null)) {
    alert (\"$l_j_select_company\");
    return false;
  } else {
    new_url = p_url + valeur;
    window.opener.location.href=new_url;
    window.close();
    return true;
  }
}

\end{verbatim}
\end{minipage}
}

% Documentation technique d'OBM : Gestion des droits et profils
% ALIACOM Pierre Baudracco
% $Id$


%%\clearpage
\subsection{Gestion des droits et profils}

La gestion des droits est le m�canisme qui permet le contr�le et l'autorisation des informations accessibles et des actions ex�cut�es par un utilisateur.\\

L'impl�mentation de la gestion des droits a �t� effectu�e avec les objectifs suivants :\\
\begin{itemize}
\item Syst�me peu intrusif dans le code (Eliminer ou �viter au maximum les tests de droits d'acc�s dans le code des modules)
\item Niveau de granularit� � l'action ex�cut�e
\item Faciliter l'�volution (modification des droits, des profils,...)
\item Syst�me l�ger et performant (si possible sans acc�s � la base de donn�es)
\item Syst�me s�r.
\end{itemize}
\vspace{0.3cm}

Les fonctionnalit�es propos�es en standard (sans code sp�cifique dans un module) :\\
\begin{itemize}
\item D�finition des sections et modules accessibles par profil.
\item Autorisation d'ex�cution au niveau de l'action d'un module par profil
\item Tout acc�s ou action non d�fini est interdit
\item Possibilit� de tests plus sp�cifiques dans un module (ex: champ affich� selon droit pr�cis) par utilisation de l'API de droits dans le module.
\item API simple de tests de droits.
\end{itemize}


\subsubsection{Droits, sections, modules et actions}

\paragraph{D�finition du droit d'acc�s � une section}

La d�finition du droit d'acc�s (visibilit�) � une section est pr�cis�e dans la d�finition des sections dans \fichier{obminclude/global\_pref.inc}
Un droit d'acc�s unitaire est requis (l'utilisateur doit avoir le droit de lecture \variable{\$cright\_read} sur la section USER pour qu'elle s'affiche dans l'exemple suivant).\\

\shadowbox{
\begin{minipage}{13cm}
\begin{verbatim}
if ($cgp_show["section"]["user"]) {
  $sections["user"] = array('Name' => $l_section_user,
                           'Url'  => $cgp_show["section"]["user"],
                           'Right'=> $cright_read);
}
\end{verbatim}
\end{minipage}
}


\paragraph{D�finition du droit d'acc�s � un module}

La d�finition du droit d'acc�s � un module est pr�cis�e dans la d�finition des modules dans \fichier{obminclude/global\_pref.inc}
Un droit d'acc�s unitaire est requis (l'utilisateur doit avoir le droit de lecture \variable{\$cright\_read} sur le module \settings pour y acc�der dans l'exemple suivant).\\

\shadowbox{
\begin{minipage}{13cm}
\begin{verbatim}
  if ($cgp_show["module"]["settings"]) {
    $modules["settings"] = array(
                               'Name'=> $l_module_settings,
                               'Ico' => "$ico_setting",
		               'Url' => "$path/settings/settings_index.php",
			       'Right'=> $cright_read);
  }
\end{verbatim}
\end{minipage}
}

\subsubsection{D�finition d'un profil}


\subsubsection{Sp�cifications de la notion de visibilit�}

Une api tr�s simple (1 fonction d�finie dans global.inc) est disponible :\\

\shadowbox{
\begin{minipage}{13cm}
\begin{verbatim}
is_entity_visible($entity, $obm_q, $uid='''')
\end{verbatim}
\end{minipage}
}
\vspace{0.4cm}



% Documentation technique d'OBM : Gestion des dates
% ALIACOM Pierre Baudracco
% $Id$


\subsection{Gestion des dates}

Les probl�mes de format des dates sont g�n�riques et pariculi�rement sensibles dans une application comme \obm fonctionnant avec diff�rentes bases de donn�es et permettant � chaque utilisateur de s�lectionner sa propre langue.\\

Afin de simplifier la gestion des dates \obm d�finit un cadre avec des r�gles d'utilisation des dates et une api.

\subsubsection{Cadre g�n�ral et r�gles d'utilisation}

\begin{tabular}{|p{7cm}|p{7cm}|}
\hline
\textbf{R�gles} & \textbf{Implications} \\
\hline
\obm r�cup�re les dates de la base de donn�es au format \textbf{Timestamp Unix}. & La BD doit savoir retourner une date au format \textbf{Timestamp Unix}.
Voir :
\begin{itemize}
\item \fonction{sql\_date\_format()}
\end{itemize}\\
\hline
\obm propose des fonctions de formattage des dates r�cup�r�es tenant compte des param�tres utilisateur.
&
Toute date r�cup�r�e de la BD doit utiliser une des fonctions de formattage avant affichage.
Voir :
\begin{itemize}
\item \fonction{date\_format()}
\item \fonction{isodate\_format()}
\item \fonction{datetime\_format()}
\end{itemize}\\
\hline
\obm re�oit les dates � ins�rer en BD au format ISO ``\textbf{AAAA-MM-JJ HH:MM:SS}''
&
La BD doit accepter en entr�e des dates au format ISO.
La fonction \fonction{calendar()} permet de g�n�rer les champs de saisie de date. Un champ de date est divis� en deux parties:
\begin{itemize}
\item Un champ de saisie
\item Un popup calendrier
\end{itemize}
La fonction javascript \fonction{live\_check\_date()} permet de convertir la date saisie au format ISO. Ces fonctions sont d�finies dans le fichier \fichier{calendar\_js.inc}
\\
\hline
OBM acceptant en simultan� des langues diff�rentes pour les utilisateurs, propose des labels pour le nom des mois, et jours de semaine.
&
Voir :
\begin{itemize}
\item \variable{\$l\_monthsofyear}
\item \variable{\$l\_monthsofyearshort}
\item \variable{\$l\_daysofweek}
\item \variable{\$l\_daysofweekshort}
\item \variable{\$l\_daysofweekfirst}
\end{itemize}\\
\hline
\end{tabular}


\subsubsection{R�cup�ration de date au format Timestamp Unix}

\paragraph{Impl�mentation multi-base de donn�es}

\fonction{sql\_date\_format()} d�finie dans le fichier \fichier{global\_query.inc} :

\shadowbox{
\begin{minipage}{13cm}
\begin{verbatim}
function sql_date_format($db_type, $field, $as="") {
  global $db_type_mysql, $db_type_pgsql;

  if ($db_type == $db_type_mysql) {
    $ret = "UNIX_TIMESTAMP($field)";
    if ($as != "") {
      $ret .= " as $as";
    }
  } elseif ($db_type == $db_type_pgsql) {
    $ret = "EXTRACT (EPOCH from $field)";
    if ($as != "") {
      $ret .= " as $as";
    }
  } else {
    $ret = $field;
  }

  return $ret;
}
\end{verbatim}
\end{minipage}
}
\paragraph{Utilisation} dans une requ�te :\\

\shadowbox{
\begin{minipage}{13cm}
\begin{verbatim}
  $obm_q = new DB_OBM;
  $db_type = $obm_q->type;
  $datealarm = sql_date_format($db_type, "deal_datealarm", "datealarm");
\end{verbatim}
\end{minipage}
}

\subsubsection{Utilisation du fichier calendar\_js.inc}
Le fichier calendar\_js.inc, disponible dans /obminclude/javascript permet la gestion des champs de saisie date ainsi que de la
popup de saisie de date.

\paragraph{Utilisation} :\\

\shadowbox{
\begin{minipage}{13cm}
\begin{verbatim}
	Syntaxe:
	<script>calendar(nom_du_champ,variable_php [,champ_�_comparer])</script>
	
	Exemple:
	<script>calendar('tf_date_begin','$datebegin', 'tf_date_end')</script>
\end{verbatim}
\end{minipage}
}

Dans cette exemple, le nom du champ de saisie sera tf\_date\_begin et sera affect� � la variable \$datebegin. \\
Le troisi�me param�tre, optionel, est le champ � laquel la date saisie sera compar�. Si la date de saisie est sup�rieur � la date compar�, cette derni�re prendra la valeur de la date saisie. \\

Lors de la saisie d'une date, la fonction \fonction{live\_check\_date()} transforme la date au format ISO si son format est valide (en fonction des pr�f�rences utilisateur) et si la date est valide. \\
Dans le cas de non validit�, l'application retourne un message d'erreur � l'utilisateur.
% Documentation technique d'OBM : Gestion du focus
% ALIACOM Pierre Baudracco
% $Id$


%%\clearpage
\subsection{Gestion du focus}

Afin d'am�liorer l'ergonomie d'utilisation d'\obm, certaines facilit�s sont disponibles pour la gestion du focus.\\

Le focus est positionn� par d�faut sur le bouton de soumission du formulaire nomm� \variable{f\_search} si existant dans la page.
Il suffit donc d'appeler le formulaire de recherche d'un module \variable{f\_search} et son bouton de validation \variable{submit} pour qu'� l'arriv�e dans ce module la pression de la touche ENTREE lance une recherche. 

% Documentation technique d'OBM : module Deal
% ALIACOM Pierre Baudracco
% $Id$


\clearpage
\section{Affaire (module \deal)}

Le module \deal d'\obm.

\subsection{Organisation de la base de donn�es}

Le module \deal utilise 6 tables :
\begin{itemize}
 \item Deal
 \item DealStatus
 \item DealType
 \item DealCategory
 \item DealCategoryLink
 \item taskType
\end{itemize}

\subsection{Deal}
Table principale des informations d'une affaire.\\

\begin{tabular}{|p{3cm}|c|p{5.4cm}|p{2.6cm}|}
\hline
\textbf{Champs} & \textbf{Type} & \textbf{Description} & \textbf{Commentaire} \\
\hline
\_id & int 8 & Identifiant & Cl� primaire \\
\hline
\_timeupdate & timestamp 14 & Date de mise � jour & \\
\hline
\_timecreate & timestamp 14 & Date de cr�ation & \\
\hline
\_userupdate & int 8 & Id du modificateur & \\
\hline
\_usercreate & int 8 & Id du cr�ateur & \\
\hline
\_number & varchar 32 & Num�ro ou identifiant de l'affaire & \\
\hline
\_label & varchar 128 & Label de l'affaire & \\
\hline
\_datebegin & date & Date de d�but de l'affaire & \\
\hline
\_parentdeal\_id & int 8 & Suraffaire d'appartenance de l'affaire & \\
\hline
\_type\_id & int 8 & Type de l'affaire & (Achat, Vente,...) \\
\hline
\_tasktype\_id & int 8 & Domaine d'activit� de l'affaire & (Int�gration,...) \\
\hline
\_company\_id & int 8 & Soci�t� en relation & \\
\hline
\_contact1\_id & int 8 & Contact 1 de l'affaire & \\
\hline
\_contact2\_id & int 8 & Contact 2 de l'affaire & \\
\hline
\_marketingmanager\_id & int 8 & Responsable commercial de l'affaire & \\
\hline
\_technicalmanager\_id & int 8 & Responsable technique de l'affaire & \\
\hline
\_dateproposal & date & Date de la derni�re proposition commerciale & \\
\hline
\_amount & Decimal(12,2) & Montant estim� ou propos� de l'affaire & \\
\hline
\_hitrate & int 3 & Pourcentage de r�ussite estim� & \\
\hline
\_status\_id & int 2 & Etat de l'affaire & \\
\hline
\_datealarm & date & Date d'alarme pour la prochaine action & \\
\hline
\_archive & char 1 & Indicateur d'archivage & (1 = 0ui)\\
\hline
\_todo & varchar 128 & Action � faire & \\
\hline
\_privacy & int 2 & Indicateur de visibilit� de l'affaire & \\
\hline
\_comment & text (64k) & Commentaire &\\
\hline
\end{tabular}


\subsection{DealStatus}
Table de r�f�rence des �tats d'une affaire.\\

\begin{tabular}{|p{3cm}|c|p{5.4cm}|p{2.6cm}|}
\hline
\textbf{Champs} & \textbf{Type} & \textbf{Description} & \textbf{Commentaire} \\
\hline
\_id & int 8 & Identifiant & Cl� primaire \\
\hline
\_project\_id & int 8 & Projet de la t�che & \\
\hline
\_timeupdate & timestamp 14 & Date de mise � jour & \\
\hline
\_timecreate & timestamp 14 & Date de cr�ation & \\
\hline
\_userupdate & int 8 & Id du modificateur & \\
\hline
\_usercreate & int 8 & Id du createur & \\
\hline
\_label & varchar 24 & Label de l'�tat & \\
\hline
\_order & int 2 & Ordre d'affichage de l'�tat & \\
\hline
\_hitrate & char 3 & Pourcentage de r�ussite associ� � l'�tat & \\
\hline
\end{tabular}

\subsubsection{Remarques}

\paragraph{dealstatus\_hitrate} : Pourcentage de r�ussite associ� � l'�tat. La s�lection de cet �tat positionne automatiquement par d�faut le pourcentage de r�ussite de l'affaire � ce pourcentage.
La valeur positionn�e dans l'affaire peut cependant �tre modifi�e.


\subsection{DealType}
Table de r�f�rence des types d'affaire.

\begin{tabular}{|p{3cm}|c|p{5.4cm}|p{2.6cm}|}
\hline
\textbf{Champs} & \textbf{Type} & \textbf{Description} & \textbf{Commentaire} \\
\hline
\_id & int 8 & Identifiant & Cl� primaire \\
\hline
\_timeupdate & timestamp 14 & Date de mise � jour & \\
\hline
\_timecreate & timestamp 14 & Date de cr�ation & \\
\hline
\_userupdate & int 8 & Id du modificateur & \\
\hline
\_usercreate & int 8 & Id du createur & \\
\hline
\_label & varchar 16 & Label du type & \\
\hline
\_inout & char 1 & Indicateur d'entr�e ou sortie du type de l'affaire & (+ = entr�e) \\
\hline
\end{tabular}

\subsubsection{Remarques}

\paragraph{dealtype\_inout} : Indicateur d'entr�e ou sortie du type d'affaire. Une affaire dont le type a cet indicateur � '+' sera compt�e positivement dans les calculs ou estimations de chiffre d'affaire.

Une affaire dont le type a l'indicateur � '-' aura son montant compt� n�gativement dans les calculs.
Cependant le module \deal �tant surtout destin� aux forces de vente, ceci ne devrait pas �tre fr�quent.


\subsection{DealCategory}
Table des cat�gories d'affaire. Une affaire peut �tre reli�e � plusieurs cat�gories.\\

\begin{tabular}{|p{3cm}|c|p{5.4cm}|p{2.6cm}|}
\hline
\textbf{Champs} & \textbf{Type} & \textbf{Description} & \textbf{Commentaire} \\
\hline
\_id & int 8 & Identifiant & Cl� primaire \\
\hline
\_timeupdate & timestamp 14 & Date de mise � jour & \\
\hline
\_timecreate & timestamp 14 & Date de cr�ation & \\
\hline
\_userupdate & int 8 & Id du modificateur & \\
\hline
\_usercreate & int 8 & Id du createur & \\
\hline
\_code & int 8 & Code de la cat�gorie & \\
\hline
\_label & varchar 100 & Label de la cat�gorie & \\
\hline
\end{tabular}


\subsubsection{DealCategoryLink}
Table de liaison entre les affaires et cat�gories d'affaire.\\

\begin{tabular}{|p{3cm}|c|p{5.4cm}|p{2.6cm}|}
\hline
\textbf{Champs} & \textbf{Type} & \textbf{Description} & \textbf{Commentaire} \\
\hline
\_category\_id & int 8 & R�f�rence � la cat�gorie & \\
\hline
\_deal\_id & int 8 & R�f�rence � l'affaire & \\
\hline
\end{tabular}


\subsection{Actions et droits}

Voici la liste des actions du module \project, avec le droit d'acc�s requis ainsi qu'une description sommaire de chacune d'entre elles.\\

\begin{tabular}{|l|c|p{9.5cm}|}
 \hline
 \textbf{Intitul�} & \textbf{Droit} & \textbf{Description} \\
 \hline
 \hline
  index & read & (D�faut) formulaire de recherche d'affaires. \\ 
 \hline
  search & read & R�sultat de recherche d'affaires. \\
 \hline
  new & write & Formulaire de cr�ation d'une affaire. \\
 \hline
  detailconsult & read & Fiche d�tail d'une affaire. \\
 \hline
  detailupdate & write & Formulaire de modification d'une affaire. \\
 \hline
  insert & write & Insertion d'une affaire. \\
 \hline
  update & write & Mise � jour d'une affaire. \\
 \hline
  quick\_update & write & Mise � jour rapide d'une affaire. \\
 \hline
  check\_delete & write & V�rification avant suppression d'une affaire. \\
 \hline
  delete & write & Suppression d'une affaire. \\
 \hline
  affect & write & Formulaire d'affectation � une suraffaire. \\
 \hline
  affect\_update & write & Affectation � une suraffaire. \\
 \hline
  admin & read\_admin & Ecran d'administration des affaires. \\
 \hline
  kind\_insert & write\_admin & Ajout d'un type d'affaire. \\
 \hline
  kind\_update & write\_admin & Modification d'un type d'affaire. \\
 \hline
  kind\_checklink & write\_admin & V�rification avant suppression d'un type d'affaire. \\
 \hline
  kind\_delete & write\_admin & Suppression d'un type d'affaire. \\
 \hline
  status\_insert & write\_admin & Ajout d'un �tat d'affaire. \\
 \hline
  status\_update & write\_admin & Modification d'un �tat d'affaire. \\
 \hline
  status\_checklink & write\_admin & V�rification avant suppression d'un �tat d'affaire. \\
 \hline
  status\_delete & write\_admin & Suppression d'un �tat d'affaire. \\
 \hline
  cat\_insert & write\_admin & Ajout d'une cat�gorie d'affaire. \\
 \hline
  cat\_update & write\_admin & Modification d'une cat�gorie d'affaire. \\
 \hline
  cat\_checklink & write\_admin & V�rification avant suppression d'une cat�gorie d'affaire. \\
 \hline
  cat\_delete & write\_admin & Suppression d'une cat�gorie d'affaire. \\
 \hline
  display & read & Ecran de modification des pr�f�rences d'affichage. \\
 \hline
  dispref\_display & read & Modifie l'affichage d'un �l�ment. \\
 \hline
  dispref\_level & read & Modifie l'ordre d'affichage d'un �l�ment. \\
 \hline
  document\_add & write & Ajout de liens vers des documents. \\
 \hline
  parent\_search & read & R�sultat de recherche de suraffaires. \\
 \hline
  parent\_new & write & Formulaire de cr�ation d'une suraffaire. \\
 \hline
  parent\_detailconsult & read & Fiche d�tail d'une suraffaire. \\
 \hline
  parent\_detailupdate & write & Formulaire de modification d'une suraffaire. \\
 \hline
  parent\_insert & write & Insertion d'une suraffaire. \\
 \hline
  parent\_update & write & Mise � jour d'une suraffaire. \\
 \hline
  parent\_delete & write & Suppression d'une suraffaire. \\
 \hline
\end{tabular}

% Documentation technique d'OBM : module List
% ALIACOM Pierre Baudracco
% $Id$


\clearpage
\section{List}

Le module \List \obm.

.\\
Todo\\
contacts statiques et dynamiques\\
2 listes affich�es\\
tests des requetes (correction + limites ressources)\\
g�n�ration de la requete : addslashes dans elements du tableau criteria\\

\subsection{Organisation de la base de donn�es}

Le module \List utilise 2 tables :
\begin{itemize}
 \item List
 \item ContactList
\end{itemize}
\vspace{0.3cm}

La table List stocke les informations g�n�rales sur les listes, la table ContactList assure la liaison des contacts statiques avec une liste.


\subsubsection{La table List}
Table principale des informations d'une liste.\\

\begin{tabular}{|p{3cm}|c|p{5.4cm}|p{2.6cm}|}
\hline
\textbf{Champs} & \textbf{Type} & \textbf{Description} & \textbf{Commentaire} \\
\hline
\_id & int 8 & Identifiant & Cl� primaire \\
\hline
\_timeupdate & timestamp 14 & Date de mise � jour & \\
\hline
\_timecreate & timestamp 14 & Date de cr�ation & \\
\hline
\_userupdate & int 8 & Id du modificateur & \\
\hline
\_usercreate & int 8 & Id du cr�ateur & \\
\hline
\_privacy & int 1 & Visibilit� de la liste & \\
\hline
\_name & varchar 64 & Nom de la liste & \\
\hline
\_subject & varchar 128 & Sujet de la liste & \\
\hline
\_email & varchar 128 & Adresse E-mail de la liste & \\
\hline
\_mailing\_ok & int 1 & Indicateur de prise en compte d'activation pour mailing & \\
\hline
\_static\_nb & int 10 & Nombre de contacts statiques (directs) associ�s � la liste & \\
\hline
\_query\_nb & int 10 & Nombre total de contacts de la liste & \\
\hline
\_query & text (64k) & Requ�te sauvegard�e des crit�res (modes normal et expert) & \\
\hline
\_structure & text (64k) & Description des crit�res du mode normal (graphique) &\\
\hline
\end{tabular}


\subsubsection{Le champ query}

Ce champ stocke la requ�te SQL g�n�r�e manuellement (mode expert) ou automatiquement en fonction des crit�res (mode normal).


\subsubsection{Le champ structure (crit�res)}

Ce champ stocke les crit�res de recherche saisis graphiquement.
A noter : le tableau des crit�res est s�rialis� pour stockage dans ce champ.


\subsubsection{La table ContactList}
Table de liaison entre une liste et ses contacts statiques.\\

\begin{tabular}{|p{3cm}|c|p{5.4cm}|p{2.6cm}|}
\hline
\textbf{Champs} & \textbf{Type} & \textbf{Description} & \textbf{Commentaire} \\
\hline
\_list\_id & int 8 & Identifiant de la liste & \\
\hline
\_contact\_id & int 8 & Identifiant du contact & \\
\hline
\end{tabular}


\subsection{D�termination mode normal ou mode expert}

Il n'y a pas de champ indicateur du mode de la liste.
Celui- ci est d�termin� en fonction du contenu des champs list\_query et list\_structure :\\

\begin{tabular}{|p{1.5cm}|p{5.5cm}|p{5.5cm}|}
\hline
\textbf{Mode} & \textbf{D�termination} & \textbf{Pr�cisions}\\
\hline
Expert & Crit�res vides et requ�te non vide & Crit�res vides ne signifie pas champ structure vide car ce champ est s�rialis�. Unserialize(structure) vide \\
\hline
Normal & Si non expert & \\
\hline
\end{tabular}


\subsubsection{API d�finie}

Une api tr�s simple (1 fonction d�finie dans list\_query.inc) est disponible :\\

\shadowbox{
\begin{minipage}{13cm}
\begin{verbatim}
function check_list_mode(&$query, &$criteria) {
\end{verbatim}
\end{minipage}
}
\begin{itemize}
 \item \textbf{\$query} : requ�te enregistr�e de la liste (list\_query). \\
 \item \textbf{\$criteria} : crit�res de recherches. Attention ce n'est pas le champ structure mais la champ struture ar�s d�serialisation.\\
\end{itemize}



\subsection{Actions et droits}

Voici la liste des actions du module \List, avec le droit d'acc�s requis ainsi qu'une description sommaire de chacune d'entre elles.\\

\begin{tabular}{|l|c|p{9.5cm}|}
 \hline
 \textbf{Intitul�} & \textbf{Droit} & \textbf{Description} \\
 \hline
 \hline
  index & read & (D�faut) formulaire de recherche de listes. \\ 
 \hline
  search & read & R�sultat de recherche. \\
 \hline
  new & write & Formulaire de cr�ation d'une liste. \\
 \hline
  new\_criterion & write & Formulaire de saisie d'un nouveau crit�re graphique. \\
 \hline
  detailconsult & read & Fiche d�tail d'une liste. \\
 \hline
  detailupdate & write & Formulaire de modification d'une liste. \\
 \hline
  insert & write & Insertion d'une liste. \\
 \hline
  update & write & Mise � jour d'une liste. \\
 \hline
  check\_delete & write & V�rification avant suppression d'une liste. \\
 \hline
  delete & write & Suppression d'une liste. \\
 \hline
  contact\_add & write & Ajouts de contacts statiques � la liste\\
 \hline
  contact\_del & write & Suppression de contacts statiques de la liste\\
 \hline
\end{tabular}

% Documentation technique d'OBM : module Agenda
% ALIACOM Mehdi Rande
% $Id$

\clearpage
\section{L'agenda partagé}

L'\agenda \obm  est un agenda partagé permettant d'inserer, modifier, 
consulter ou supprimer des évenements pour un ou plusieurs utilisateurs 
simultanés.

\subsection{Organisation de la base de données}

Le \calendar utilise 5 tables :
\begin{itemize}
 \item CalendarEvent
 \item CalendarException
 \item CalendarCategory
 \item EntityEvent
 \item EntityRight
\end{itemize}

\subsubsection{CalendarCategory}
Cette table est utilisée pour sotcker les catégories des événements.\\

\begin{tabular}{|p{3cm}|c|p{5.4cm}|p{2.6cm}|}
\hline
\textbf{Champs} & \textbf{Type} & \textbf{Description} & \textbf{Commentaire} \\
\hline
\_id & int 8 & Identifiant & Clé primaire \\
\hline
\_timeupdate & timestamp 14 & Date de mise à jour & \\
\hline
\_timecreate & timestamp 14 & Date de création & \\
\hline
\_userupdate & int 8 & Id du modificateur & \\
\hline
\_usercreate & int 8 & Id du créateur & \\
\hline
\_label & varchar 128 &  Label de la catégorie & \\
\hline
\end{tabular}

\subsubsection{CalendarEvent}
Cette table stocke toute la description d'un évenement ainsi que ces caractéristiques.\\

\begin{tabular}{|p{3cm}|c|p{5.4cm}|p{2.6cm}|}
\hline
\textbf{Champs} & \textbf{Type} & \textbf{Description} & \textbf{Commentaire} \\
\hline
\_id & int 8 & Identifiant &  Clé primaire \\
\hline
\_timeupdate & timestamp 14 & Date de mise à jour & \\
\hline
\_timecreate & timestamp 14 & Date de création & \\
\hline
\_userupdate & int 8 & Id du modificateur &  Clé etrangère\\
\hline
\_usercreate & int 8 & Id du créateur & Clé etrangère\\
\hline
\_owner & int 8 & Id du propriétaire & Clé etrangère\\
\hline
\_title & varchar 255 & Titre & \\
\hline
\_location & varchar 100 & Lieu & \\
\hline
\_description & text & Description & \\
\hline
\_category\_id & int 8 & Id de la catégorie & Clé etrangère\\
\hline
\_priority & int 2 & Priorité : <1> Basse <2> Normal <3> Haute & \\
\hline 
\_privacy & int 2 & Privé : <0> Non <1> Oui & \\
\hline
\_date & timestamp & Date de début de la première occurence de l'événement & \\
\hline
\_duration & int 8 & Durée de l'événement en secondes & \\
\hline
\_allday & int 1 & Evénement sur toute la journée <0> Non <1> Oui & \\
\hline
\_repeatkind & varchar 20 & Type de répétition & \\
\hline
\_repeatfrequence & int 3 & Fréquence de répétition & \\
\hline
\_repeatdays & varchar 7 & Jours de répétion pour les répétition de type
hebdomadaire & \\
\hline
\_endrepeat & timestamp & Date de fin de répétition & \\
\hline
\end{tabular}


\subsubsection{CalendarException}

Cette table stocke les exceptions des événements périodiques.\\

\begin{tabular}{|p{3cm}|c|p{5.4cm}|p{2.6cm}|}
\hline
\textbf{Champs} & \textbf{Type} & \textbf{Description} & \textbf{Commentaire} \\
\hline
\_timeupdate & timestamp 14 & Date de mise à jour & \\
\hline
\_timecreate & timestamp 14 & Date de création & \\
\hline
\_userupdate & int 8 & Id du modificateur & Clé etrangère\\
\hline
\_usercreate & int 8 & Id du créateur & Clé etrangère\\
\hline
\_event\_id & int 8 & Id de l'événement & Clé etrangère - Clé primaire \\
\hline
\_date & timestamp & Date de l'exception & \\
\hline
\end{tabular}


\subsubsection{EventEntity}

Cette table stocke toutes les relations entre les événements et les entités (utilisateurs, groupes, resources,...).\\


\begin{tabular}{|p{3cm}|c|p{5.4cm}|p{2.6cm}|}
\hline
\textbf{Champs} & \textbf{Type} & \textbf{Description} & \textbf{Commentaire} \\
\hline
\_timeupdate & timestamp 14 & Date de mise à jour & \\
\hline
\_timecreate & timestamp 14 & Date de création & \\
\hline
\_userupdate & int 8 & Id du modificateur & Clé etrangère\\
\hline
\_usercreate & int 8 & Id du créateur & Clé etrangère\\
\hline
\_event\_id & int 8 & Id de l'événement & Clé etrangère - Clé primaire \\
\hline
\_entity\_id & int 8 & Id de l'entité & Clé etrangère - Clé primaire \\
\hline
\_entity & varchar 32 & nom de l'entité & Clé primaire - 'user', 'resource',... \\
\hline
\_state & char 1 & Etat de l'événément : <R> Refusé, <W> En attente, et <A> 
Accepté  & \\
\hline 
\_required & int 1 & Détermine si l'utilisateur est nécessaire pour l'événement
: <0> Non <1> Oui & \\
\hline
\end{tabular}


\subsubsection{EntityRight}

Table générique de gestion des droits individuels ou ACL.
L'entité est l'objet sur lequel on fixe les droits (un calendrier, une ressource,...).
Le consommateur est l'objet à qui on donne les droits (utilisateur ou groupe).

Voir section \ref{acl}.\\

\begin{tabular}{|p{3cm}|c|p{5.4cm}|p{2.6cm}|}
\hline
\textbf{Champs} & \textbf{Type} & \textbf{Description} & \textbf{Commentaire} \\
\hline
\_entity &varchar 32 & type de l'entité & Clé primaire - 'calendar', 'resource'\\
\hline
\_entity\_id & int 8 & Id de l'entité & Clé primaire\\
\hline
\_consumer & varchar 32 & type du consommateur & Clé primaire - 'user, 'group\\
\hline
\_consumer\_id & int 8 & Id du consommateur & Clé primaire\\
\hline 
\_read & int 1 & Droit de lecture <0> Non <1> Oui  & \\
\hline
\_write & int 1 & Droit d'ecriture <0> Non <1> Oui & \\
\hline
\end{tabular}


\subsection{Actions et droits}

Voici la liste des actions du module \agenda, avec le droit d'accès requis 
ainsi qu'une description sommaire de chacune d'entre elles.\\

\begin{tabular}{|l|c|p{9.5cm}|}
 \hline
 \textbf{Intitulé} & \textbf{Droit} & \textbf{Description} \\
 \hline
  index & read & Accueil \\ 
 \hline
  decision & write & Définit l'état d'une liste d'événements \\
 \hline
  view\_year & read & Vue annuelle \\
 \hline
  view\_month & read & Vue mensuelle \\
 \hline
  view\_week & read & Vue hebdomadaire \\
 \hline
  view\_day & read & Vue quotidienne \\
 \hline
  new & write & Formulaire de nouvel événement \\
 \hline
  insert & write & Insertion de l'événement \\
 \hline
  detail\_consult & read & Consultation d'un événement \\
 \hline
  check\_delete & admin & Confirmation de suppression d'événement \\
 \hline
  delete & write & Suppression d'un événement \\
 \hline
  detailupdate & write & Formulaire de mise à jour d'un événement \\
 \hline
  update & write & Mise à jour d'un événement \\
 \hline
  update\_decision & write & Mise à jour de l'état d'un événements \\
 \hline
  rights\_admin & write & Formulaire de mise à jour des droits \\
 \hline
  rights\_update &  write & Mise à jour des droits \\
 \hline
  new\_meeting & write & Formulaire de nouvelle réunion \\
 \hline
  perform\_meeting & write & Vue réunion \\
 \hline
\end{tabular}


\subsection{Paramètres}

Différents paramètres concernant les utilisateurs, groupes et leur stockage dans le tableau de paramètres :\\

\begin{tabular}{|l|p{3cm}|p{8cm}|}
 \hline
 \textbf{Haschage \$agenda} & \textbf{Paramètre} & \textbf{Description} \\
 \hline
  [``user\_id''] & param\_user & Utilisateur spécifié lors de l'acceptation d'événement (déléguation) \\ 
 \hline
  [``group\_view''] & param\_group & Groupe sélectionné pour l'affichage du calendrier d'un groupe \\ 
 \hline
  [``new\_group''] & new\_group & Indicateur de sélection d'un nouveau groupe\\
 \hline
  [``sel\_group\_id''] & sel\_group\_id & Groupes sélectionnés (formulaire rdv) \\ 
 \hline
  [``sel\_user\_id''] & sel\_user\_id ou sel\_ent[``user''] & Utilisateurs sélectionnés \\ 
 \hline
  [``sel\_resource\_id''] & sel\_resource\_id ou sel\_ent[``resource] & Ressources sélectionnées \\ 
 \hline
\end{tabular}


\subsection{Sélection et stockage des utilisateurs, resources, groupes}

Lors de la sélection de plusieurs utilisateurs ou entités, ceux-ci sont gardés en session afin de maintenir la sélection en changeant de vue ou en se déplaçant dans le calendrier.
De même quand un événement est créé pour plusieurs utilisateurs, la vue passe automatiquement en vue multi-utilisateurs avec ces utilisateurs sélectionnés.

Cependant les données stockées en session ne sont pas systématiquement les données sélectionnées; exemple :
\begin{itemize}
\item Quand un groupe est sélectionné, la vue passe automatiquement en vue membres du groupe, et le choix des utilisateurs est limité au groupe (restreint à ceux qui ont donné les droits en lecture).
\item La sélection est limitée par défaut au nombre d'utilisateurs affichables simultanément (par défaut 6).
\end{itemize}
\vspace{0.3cm}

Règles utilisées pour la sélection et le stockage en session des entités :\\

Les entités sélectionnées sont stockées dans le tableau \variable{\$cal\_entity\_id} qui est stocké en session à la fin de l'exécution (car la sélection peut être modifiée par les traitements, comme la restriction au nombre d'utilisateurs affichables simultanément) et qui comprend plusieurs entrées :
\begin{itemize}
\item[-] [user] : tableau d'id d'utilisateurs
\item[-] [resource] : tableau d'id de ressources
\item[-] [group] : tableau d'id de groupes
\item[-] [group\_view] : groupe sélectionné pour l'affichage des utilisateurs
\end{itemize}
\vspace{0.3cm}

Le tableau \variable{\$cal\_entity\_id} est copié dans le hashage global \variable{\$agenda[entity]} afin de le mettre directement à disposition des différentes fonctions.\\

\#nb correspond au nombre maximal de calendriers affichables simultanément.\\

Si aucune entité n'est sélectionnée (utilisateur, groupe, ressource) l'utilisateur connecté est automatiquement sélectionné.\\

\begin{longtable}{|p{4cm}|p{10cm}|}
\hline
\textbf{Action} & \textbf{Sélection} \\

\hline
Affichage d'une vue
&
\begin{itemize}
\item Session inchangée (listes déjà été limitées à \#nb)
\item Affichage calendriers  : contenu de [user], [resource]
\item Group sélectionné : inchangé
\end{itemize}
\\ 

\hline
Sélection de 1 ou plusieurs utilisateurs ou ressources en mode vue
&
\begin{itemize}
\item[-] [user] = sélection d'utilisateurs limitée à \#nb
\item[-] [group] = inchangé
\item[-] [resource] = sélection de ressources limitée à \#nb - \#nb\_users
\item Affichage calendriers : contenu de [user] [resource]
\item Groupe sélectionné : aucun 
\end{itemize}
\\ 

\hline
Sélection de 1 groupe mode vue
&
\begin{itemize}
\item[-] [user] = utilisateurs du groupe sélectionné limité à \#nb
\item[-] [group] = groupe sélectionné
\item[-] [resource] = vide
\item Affichage calendriers : contenu de [user]
\item Group sélectionné : le groupe sélectionné
\end{itemize}
\\ 

\hline
Fenêtre nouveau rdv
&
\begin{itemize}
\item Session inchangée (listes déjà été limitées à \#nb)
\item Utilisateurs pré-sélectionnés : contenu de [user] 
\item Groupes pré-sélectionnés : contenu de [group] 
\item Ressources pré-sélectionnées : contenu de [resource]
\end{itemize}
\\ 

\hline
Création rdv : Sélection de 1 ou plusieurs utilisateurs, groupes, ressources
&
\begin{itemize}
\item[Note] Les vues limitent les sélections, non les insertions
\item[-] [user] = utilisateurs sélectionnés, limité à \#nb
\item[-] [group] = groupes sélectionnés
\item[-] [resource] = ressources sélectionnées limité à \#nb - \#nb\_users
\item Affichage calendriers  : contenu de [user], [resource] 
\item Group sélectionné : aucun
\end{itemize}
\\

\hline
Fenêtre nouvelle réunion
&
\begin{itemize}
\item Utilisateurs pré-sélectionnés : contenu de [user] 
\item Groupes pré-sélectionnés : contenu de [group] 
\item Ressources pré-sélectionnées : contenu de [resource] 
\end{itemize}
\\ 

\hline
Recherche crénneaux disponibles
&
\begin{itemize}
\item[-] [user] = utilisateurs sélectionnés
\item[-] [group] = groupes sélectionnés
\item[-] [resource] = ressources sélectionnées
\end{itemize}
\\
\hline
\end{longtable}



\subsubsection{Remarques}
 Par défaut l'action \textbf{index} affiche la vue hebdomadaire de la semaine
 courante. Cependant si l'utilisateur courant ou bien un utilisateur qui lui a 
 cédé les droits d'ecriture sur son agenda à des événements en attente cette
 action affichera la liste des événements en attente.

\subsection{Principe de fonctionnement d'affichage d'une vue}
 Les etapes de construction de la vue sont :
 \begin{itemize}
  \item{Récupération des informations en base de données}
  \item{Traitement des données et construction du modèle}
  \item{Traitement du modèle et construction de la vue}
  \item{Affichage de la vue}
 \end{itemize}

\subsubsection{Structure du modèle}
 Le modèle de donnée est un des rare éléments à utiliser objets au sein d'un
 module. Ceci vient du fait que l'approche objet est particulierement adapté aux
 traitement à effectuer.

 Voici les différents éléments de ce modèle :
\paragraph{L'objet Event}
 L'objet Event représente un événement au sens abstrait.
 C'est à dire que quelque soit le nombre d'occurence d'un événement (dans le cas
 d'un événement répétitif) il n'y aura qu'un seul objet Event.

 \begin{tabular}{|l|p{9.5cm}|}
  \hline
   \textbf{Attribut} & \textbf{Description} \\
  \hline
   id & Identifiant \\
  \hline
   duration & Durée en secondes \\
  \hline
   title & Titre \\
  \hline
   category & Label de la catégorie \\
  \hline
   privacy & Si événement privé \\
  \hline
   description & Description de l'événement \\
  \hline
 \end{tabular}

\paragraph{L'objet Day}
 L'objet Day represente un jour. Il stockera toutes les informations relatives à
 ce jour : date, événements, ... 
 \\
 \begin{tabular}{|l|p{9.5cm}|}
  \hline
   \textbf{Attribut} & \textbf{Description} \\
  \hline
   day & Date du jour \\
  \hline
   events & Tableau d'objet Events lié avec des id utilisateurs et une heure de
   début \\
  \hline
   title & Titre \\
  \hline
 \end{tabular}
 \\
 \\
 \\
 \begin{tabular}{|l|c|p{9.5cm}|}
  \hline
   \textbf{Fonction} & \textbf{Paramètres} & \textbf{Description} \\
  \hline
   is\_same\_day & date & Return true si la date passée en parametre est la meme
   que celle de l'objet\\
  \hline
   add\_even & event - begin\_date - uid & Ajoute un événement au tableau
   \textbf{events}, le lie avec l'heure de début et l'uid passé en paramètre. Si
   l'événement été déjà stocké, l'événement est juste lié à l'uid.\\
  \hline
   get\_events & uid & Retourne tout les événement du tableau \textbf{events} 
   de l'utilisateurs dont l'uid est passé en paramètre.\\
  \hline
  \hline
   have\_events\_between & start - end & Retourne tout les événement du tableau \textbf{events} 
   qui se débute entre l'heure \textbf{start} et l'heure \textbf{end} \\
  \hline
   get\_events\_between & start - end - uid & Retourne tout les événement du tableau \textbf{events} 
   qui se débute entre l'heure \textbf{start} et l'heure \textbf{end} pour
   l'utilisateur \textbf{uid} \\
  \hline
 \end{tabular}
  \textbf{Remarques :} \\
  Les fonctions \textbf{have\_events\_between} et \textbf{get\_events\_between}
  retourn \textbf{NULL} si aucun événements ne débute durant l'interval de
  temps et que aucun événements ne se déroule durant l'interval de
  temps, \textbf{-1} si aucun événements ne débute durant l'interval de
  temps mais que des événements sont déjà en cours.
  
 \subsubsection{Affichage de la vue}
   La fonction qui doit construire est afficher la vue reçoit donc un tableau
   d'objets \textbf{Day} qui correspondent aux jours à afficher.
   Pour construire la vue, la fonction interroge juste les objets \textbf{Day}
   du tableau pour savoir si un jour donné contient des informations à afficher,
   et si oui quelles sont ces information.


\input{technique/t_mod_time.tex}
\input{technique/t_mod_project.tex}
% Documentation technique d'OBM : module Document
% ALIACOM Pierre Baudracco
% $Id$


\clearpage
\section{Document}

Le module \doc \obm.

\subsection{Organisation de la base de donn�es}

Le module \doc utilise 5 tables :
\begin{itemize}
 \item Document
 \item DocumentEntity
 \item DocumentMimeType
 \item DocumentCategory1
 \item DocumentCategory2
\end{itemize}

\subsection{Document}
Table principale des informations d'un document.\\

\begin{tabular}{|p{3cm}|c|p{5.4cm}|p{2.6cm}|}
\hline
\textbf{Champs} & \textbf{Type} & \textbf{Description} & \textbf{Commentaire} \\
\hline
\_id & int 8 & Identifiant & Cl� primaire \\
\hline
\_timeupdate & timestamp 14 & Date de mise � jour & \\
\hline
\_timecreate & timestamp 14 & Date de cr�ation & \\
\hline
\_userupdate & int 8 & Id du modificateur & \\
\hline
\_usercreate & int 8 & Id du cr�ateur & \\
\hline
\_title & varchar 255 & Titre du document & Description\\
\hline
\_name & varchar 255 & Nom du document & Nom du fichier\\
\hline
\_kind & int 8 & Type de document & (r�pertoire, fichier, lien)\\
\hline
\_mymetype\_id & int 8 & Type Mime du fichier & \\
\hline
\_category1\_id & int 8 & Cat�gorie 1 &\\
\hline
\_category2\_id & int 8 & Cat�gorie 2 &\\
\hline
\_privacy & int 1 & Visibilit� du Document & Voir aussi \_acl\\
\hline
\_size & int 15 & Taille du document & en octets \\
\hline
\_author & varchar 255 & Auteur du document & \\
\hline
\_path & text (64k) & Chemin du document &\\
\hline
\_acl & text (64k) & Liste de contr�le d'acc�s &\\
\hline
\end{tabular}

\subsubsection{Le champ kind (type de document)}

\begin{tabular}{|c|c|l|}
\hline
\textbf{variable} & \textbf{valeur} & \textbf{Description}\\
\hline
\$cdoc\_kind\_dir & 0 & R�pertoire \\
\hline
\$cdoc\_kind\_file & 1 & Fichier \\
\hline
\$cdoc\_kind\_link & 2 & Lien (http, https) \\
\hline
\end{tabular}


\subsection{DocumentEntity}

Table de liaison entre les documents et diff�rentes entit�s d'\obm.\\

\begin{tabular}{|p{3cm}|c|p{5.4cm}|p{2.6cm}|}
\hline
\textbf{Champs} & \textbf{Type} & \textbf{Description} & \textbf{Commentaire} \\
\hline
\_document\_id & int 8 & Identifiant du document & \\
\hline
\_entity\_id & int 8 & Identifiant de l'entit� & \\
\hline
\_entity & varchar 255 & Entit� (company, contact,...) & \\
\hline
\end{tabular}


\subsection{DocumentMimeType}
Table des informations de types MIME.\\

\begin{tabular}{|p{3cm}|c|p{5.4cm}|p{2.6cm}|}
\hline
\textbf{Champs} & \textbf{Type} & \textbf{Description} & \textbf{Commentaire} \\
\hline
\_id & int 8 & Identifiant & Cl� primaire \\
\hline
\_timeupdate & timestamp 14 & Date de mise � jour & \\
\hline
\_timecreate & timestamp 14 & Date de cr�ation & \\
\hline
\_userupdate & int 8 & Id du modificateur & \\
\hline
\_usercreate & int 8 & Id du cr�ateur & \\
\hline
\_label & varchar 255 & Label du type MIME & \\
\hline
\_extension & varchar 10 & Extension du type (jpg, gif, txt,..) & \\
\hline
\_mime & varchar 255 & Type MIME \\
\hline
\end{tabular}


\subsection{DocumentCategory1}
Table des informations de cat�gorisation des documents :  category 1.
Lien monovalu�.\\

\begin{tabular}{|p{3cm}|c|p{5.4cm}|p{2.6cm}|}
\hline
\textbf{Champs} & \textbf{Type} & \textbf{Description} & \textbf{Commentaire} \\
\hline
\_id & int 8 & Identifiant & Cl� primaire \\
\hline
\_timeupdate & timestamp 14 & Date de mise � jour & \\
\hline
\_timecreate & timestamp 14 & Date de cr�ation & \\
\hline
\_userupdate & int 8 & Id du modificateur & \\
\hline
\_usercreate & int 8 & Id du cr�ateur & \\
\hline
\_label & varchar 255 & Label de la cat�gorie & \\
\hline
\end{tabular}


\subsection{DocumentCategory2}
Table des informations de cat�gorisation des documents :  category 2.
Lien monovalu�.\\

\begin{tabular}{|p{3cm}|c|p{5.4cm}|p{2.6cm}|}
\hline
\textbf{Champs} & \textbf{Type} & \textbf{Description} & \textbf{Commentaire} \\
\hline
\_id & int 8 & Identifiant & Cl� primaire \\
\hline
\_timeupdate & timestamp 14 & Date de mise � jour & \\
\hline
\_timecreate & timestamp 14 & Date de cr�ation & \\
\hline
\_userupdate & int 8 & Id du modificateur & \\
\hline
\_usercreate & int 8 & Id du cr�ateur & \\
\hline
\_label & varchar 255 & Label de la cat�gorie & \\
\hline
\end{tabular}


\subsection{Gestion du d�pot de documents}

\subsubsection{Principe}

Les informations d'un document sont stock�es en base de donn�es.
Les documents eux-m�me sont stock�s sur disque.\\

\obm g�re de fa�on autonome le d�p�t de documents; \obm g�re automatiquement :
\begin{itemize}
\item L'arborescence physique des documents
\item Le nommage et l'emplacement des documents sur disque
\end{itemize}
\vspace{0.3cm}

Ceci perm�t d'�viter :
\begin{itemize}
\item Les probl�mes de s�curit� d'acc�s aux fichiers du serveur 
\item Les probl�mes de nommage des documents (jeux de caract�res, caract�res sp�ciaux,...) sur disque
\end{itemize}


\subsubsection{Impl�mentation et gestion par d�faut}

La racine de l'arborescence des documents sur le serveur est sp�cifi�e par le param�tre de configuration \variable{\$cdocument\_root}.\\

Par d�faut \obm cr�� 9 r�pertoires nomm�s 1, 2, 3, 4, 5, 6, 7, 8 et 9 � la racine du d�p�t de documents.

\paragraph{Le chemin d'un document} est calcul� en fonction de l'id du document. Le dernier chiffre de l'identifiant du document d�finit le r�pertoire de stockage du document.
Ceci permet de r�partir les documents dans plusieurs r�peroires.

\paragraph{Le nom d'un document} est �gal � son id. Ceci �vite tout probl�me de nommage des fichiers.

\paragraph{Exemples} :
\begin{tabular}{|l|c|l|}
 \hline
 \textbf{Id Document} & \textbf{Chemin relatif} & \textbf{nom physique} \\
 \hline
  123 & 3 & \$cdocument\_root/3/123 \\ 
 \hline
  4 & 4 & \$cdocument\_root/4/4 \\ 
 \hline
  12156 & 6 & \$cdocument\_root/6/12156 \\ 
 \hline
\end{tabular}


\subsubsection{Outil de v�rification de coh�rence}

\obm fournit un outil de v�rification de coh�rence entre la base de donn�es et les fichiers dans le d�p�t de documents.\\

\textit{ADMINISTRATION -> Donn�es -> action : show\_data, module document}.\\

Cet Outil r�f�rence :
\begin{itemize}
\item Les fichiers enregistr�s en base de donn�es et non pr�sents sur disque
\item Les fichiers pr�sents sur disque sans correspondance en base de donn�es
\end{itemize}


\subsubsection{Modification du stockage des documents}

\obm permet une modification simple de la gestion du d�p�t des documents. Une fonction d�di�e est utilis�e pour calculer le chemin d'un fichier en fonction de son Id. Cette fonction est ais�ment modifiable.\\

\shadowbox{
\begin{minipage}{13cm}
\begin{verbatim}
function get_document_disk_path($id) {
\end{verbatim}
\end{minipage}
}
Fonction de calcul du chemin physique r�el d'un document sur disque.\\

Il est par exemple possible de d�finir 99 r�pertoires initiaux (� la place des 9) pour mieux supporter de gros volumes de documents.


\subsubsection{Gestion des liens}

Les liens sont stock�s uniquement en base de donn�es sans stockage physique sur disque.


\subsection{Gestion des types MIME}

A la cr�ation ou modification d'un document, le type MIME peut �tre s�lectionn� ou laiss� par d�faut.
Si le type MIME est laiss� par d�faut, \obm essaie de d�terminer automatiquement le type MIME associ� au fichier.

Ceci s'effectue en fonction de l'extention du fichier par exemple.\\

\shadowbox{
\begin{minipage}{13cm}
\begin{verbatim}
function get_auto_mime_type($document) {
\end{verbatim}
\end{minipage}
}
Fonction de d�termination automatique du type MIME d'un document.\\

Si aucun type MIME n'est d�termin� (en fonction des types MIME enregistr�s en base), le type MIME par d�faut est s�lectionn� selon le param�tre de configuration \variable{\$default\_mime}.


\subsection{Param�tres de configuration du module \doc}

\begin{tabular}{|l|l|l|}
\hline
\textbf{Param�tre} & \textbf{par d�faut} & \textbf{Description}\\
\hline
\$cdocument\_root & /var/www/obmdocuments & Racine du d�p�t de document sur disque \\
\hline
\$default\_mime & application/octet-stream & Type MIME par d�faut d'un fichier quand non d�termin� \\
\hline
\end{tabular}


\subsection{Actions et droits}

Voici la liste des actions du module \project, avec le droit d'acc�s requis ainsi qu'une description sommaire de chacune d'entre elles.\\

\begin{tabular}{|l|c|p{9.5cm}|}
 \hline
 \textbf{Intitul�} & \textbf{Droit} & \textbf{Description} \\
 \hline
 \hline
  index & read & (D�faut) formulaire de recherche de documents. \\ 
 \hline
  search & read & R�sultat de recherche de documents. \\
 \hline
  new & write & Formulaire de cr�ation d'un document. \\
 \hline
  new\_dir & write & Formulaire de cr�ation d'un r�pertoire. \\
 \hline
  tree & read & Visualisation en arborescence. \\
 \hline
  detailconsult & read & Fiche d�tail d'un document. \\
 \hline
  detailupdate & write & Formulaire de modification d'un document. \\
 \hline
  insert & write & Insertion d'un document. \\
 \hline
  insert\_dir & write & Insertion d'un r�pertoire. \\
 \hline
  update & write & Mise � jour du document. \\
 \hline
  check\_delete & write & V�rification avant suppression du document. \\
 \hline
  delete & write & Suppression du document. \\
 \hline
  dir\_check\_delete & write & V�rification avant suppression du r�pertoire. \\
 \hline
  dir\_delete & write & Suppression du r�pertoire. \\
 \hline
  admin & write & Ecran d'administartion (gestion des cat�gories). \\
 \hline
  cat1\_insert & write\_admin & Insertion de la cat�gorie 1. \\
 \hline
  cat1\_update & write\_admin & Modification de la cat�gorie 1. \\
 \hline
  cat1\_checklink & write\_admin & V�rification des liens de la cat�gorie 1. \\
 \hline
  cat1\_delete & write\_admin & Suppression de la cat�gorie 1. \\
 \hline
  cat2\_insert & write\_admin & Insertion de la cat�gorie 2. \\
 \hline
  cat2\_update & write\_admin & Modification de la cat�gorie 2. \\
 \hline
  cat2\_checklink & write\_admin & V�rification des liens de la cat�gorie 2. \\
 \hline
  cat2\_delete & write\_admin & Suppression de la cat�gorie 2. \\
 \hline
  mime\_insert & write\_admin & Insertion du type MIME. \\
 \hline
  mime\_update & write\_admin & Modification du type MIME. \\
 \hline
  mime\_checklink & write\_admin & V�rification des liens du type MIME. \\
 \hline
  mime\_delete & write\_admin & Suppression du type MIME. \\
 \hline
  display & read & Ecran de modification des pr�f�rences d'affichage. \\
 \hline
  dispref\_display & read & Modifie l'affichage d'un �l�ment. \\
 \hline
  dispref\_level & read & Modifie l'ordre d'affichage d'un �l�ment. \\
 \hline
 \hline
  ext\_get\_path & read & Appel externe pour s�lection d'un chemin. \\
 \hline
  ext\_get\_ids & read & Appel externe pour s�lection de documents. \\
 \hline
\end{tabular}

% Documentation technique d'OBM : module Contrat
% ALIACOM Pierre Baudracco
% $Id$


\clearpage
\section{Contrat}

r�vision : \obm 2.0
Le module \contract \obm.


\subsection{Organisation de la base de donn�es}

Le module \contract utilise 4 tables :
\begin{itemize}
 \item Contrat
 \item ContractPriority
 \item ContractStatus
 \item ContractType
\end{itemize}


\subsection{Contract}
Table principale des informations d'un contrat.\\

\begin{tabular}{|p{3cm}|c|p{5.4cm}|p{2.6cm}|}
\hline
\textbf{Champs} & \textbf{Type} & \textbf{Description} & \textbf{Commentaire} \\
\hline
\_id & int 8 & Identifiant & Cl� primaire \\
\hline
\_timeupdate & timestamp 14 & Date de mise � jour & \\
\hline
\_timecreate & timestamp 14 & Date de cr�ation & \\
\hline
\_userupdate & int 8 & Id du modificateur & \\
\hline
\_usercreate & int 8 & Id du cr�ateur & \\
\hline
\_deal\_id & int 8 & Affaire li�e au contrat & Facultatif \\
\hline
\_company\_id & int 8 & Soci�t� contractante & Client ou fournisseur \\
\hline
\_label & varchar 128 & Nom du contrat & \\
\hline
\_number & varchar 20 & Num�ro ou r�f�rence du contrat & \\
\hline
\_datesignature & date & Date de signature & \\
\hline
\_datebegin & date & Date de d�but & \\
\hline
\_dateexp & date & Date d'expiration & \\
\hline
\_daterenew & date & Date de renouvellement & \\
\hline
\_datecancel & date & Date de r�siliation & Aspect client \\
\hline
\_type\_id & int 8 & Type de contrat & \\
\hline
\_priority\_id & int 8 & Priorit� du contrat & \\
\hline
\_status\_id & int 8 & Etat du contrat & (En cours, termin�)\\
\hline
\_kind & int 2 & Indicateur : Client ou fournisseur & \$cck\_customer=0, \$cck\_supplier=1\\
\hline
\_format & int 2 & Indicateur : par p�riode, dur�e ou coupons & \$ccf\_period=0, \$ccf\_duration=1, \$ccf\_ticket=2\\
\hline
\_ticketnumber & int 8 & Nombre de coupons & (cas contrat par coupons)\\
\hline
\_duration & int 8 & Dur�e du contrat (en Heure) & (cas contrat par dur�e)\\
\hline
\_autorenewal & int 2 & Indicateur de renouvellement automatique & 1 = oui\\
\hline
\_contact1\_id & int 8 & Contact 1 chez le contractant &\\
\hline
\_contact2\_id & int 8 & Contact 2 chez le contractant &\\
\hline
\_techmanager\_id & int 8 & Responsable technique interne &\\
\hline
\_marketmanager\_id & int 8 & Responsable commercial interne &\\
\hline
\_privacy & int 2 & Visibilit� du contrat & 0=public, 1=priv�\\
\hline
\_archive & int 1 & Indicateur d'archivage & (1 = 0ui)\\
\hline
\_clause & text (64k) & Clauses particuli�res &\\
\hline
\_comment & text (64k) & Commentaire &\\
\hline
\end{tabular}
	 
	
\subsection{ContractType}

Table de cat�gorisation de contrats (table de r�f�rence).

\begin{tabular}{|p{3cm}|c|p{5.4cm}|p{2.6cm}|}
\hline
\textbf{Champs} & \textbf{Type} & \textbf{Description} & \textbf{Commentaire} \\
\hline
\_id & int 8 & Identifiant & Cl� primaire \\
\hline
\_timeupdate & timestamp 14 & Date de mise � jour & \\
\hline
\_timecreate & timestamp 14 & Date de cr�ation & \\
\hline
\_userupdate & int 8 & Id du modificateur & \\
\hline
\_usercreate & int 8 & Id du cr�ateur & \\
\hline
\_label & varchar 40 & Label & \\
\hline
\end{tabular}


\subsection{ContractPriority}

Table des informations de priorit� des contrats (table de r�f�rence).\\

\begin{tabular}{|p{3cm}|c|p{5.4cm}|p{2.6cm}|}
\hline
\textbf{Champs} & \textbf{Type} & \textbf{Description} & \textbf{Commentaire} \\
\hline
\_id & int 8 & Identifiant & Cl� primaire \\
\hline
\_timeupdate & timestamp 14 & Date de mise � jour & \\
\hline
\_timecreate & timestamp 14 & Date de cr�ation & \\
\hline
\_userupdate & int 8 & Id du modificateur & \\
\hline
\_usercreate & int 8 & Id du cr�ateur & \\
\hline
\_color & varchar 6 & Code couleur associ� � la priorit� \\
\hline
\_order & int 2 & Ordre d'affichage & \\
\hline
\_label & varchar 32 & Label & \\
\hline
\end{tabular}


\subsection{ContractStatus}

Table des �tats des contrats (table de r�f�rence).\\

\begin{tabular}{|p{3cm}|c|p{5.4cm}|p{2.6cm}|}
\hline
\textbf{Champs} & \textbf{Type} & \textbf{Description} & \textbf{Commentaire} \\
\hline
\_id & int 8 & Identifiant & Cl� primaire \\
\hline
\_timeupdate & timestamp 14 & Date de mise � jour & \\
\hline
\_timecreate & timestamp 14 & Date de cr�ation & \\
\hline
\_userupdate & int 8 & Id du modificateur & \\
\hline
\_usercreate & int 8 & Id du cr�ateur & \\
\hline
\_order & int 2 & Order & \\
\hline
\_label & varchar 32 & Label \\
\hline
\end{tabular}


\subsection{Actions et droits}

Voici la liste des actions du module \contract, avec le droit d'acc�s requis ainsi qu'une description sommaire de chacune d'entre elles.\\

\begin{tabular}{|l|c|p{9.5cm}|}
 \hline
 \textbf{Intitul�} & \textbf{Droit} & \textbf{Description} \\
 \hline
 \hline
  index & read & (D�faut) formulaire de recherche de contrats. \\ 
 \hline
  search & read & R�sultat de recherche de contrats. \\
 \hline
  new & write & Formulaire de cr�ation d'un contrat. \\
 \hline
  detailconsult & read & Fiche d�tail d'un contrat. \\
 \hline
  detailupdate & write & Formulaire de modification d'un contrat. \\
 \hline
  insert & write & Insertion d'un contrat. \\
 \hline
  update & write & Mise � jour du contrat. \\
 \hline
  check\_delete & write & V�rification avant suppression du contrat. \\
 \hline
  delete & write & Suppression du contrat. \\
 \hline
  priorite & write & Liste des priorites d�finies et formulaire de nouvelle priorite. \\
 \hline
  priorite\_add & write & Ajout d'une priorite . \\
 \hline
  priorite\_update & write & Modification d'une priorite. \\
 \hline
  priorite\_del & write & Suppression d'une priorite. \\
 \hline
  status & write & Liste des etats d�finies et formulaire de nouveau etat. \\
 \hline
  status\_add & write & Ajout d'un etat . \\
 \hline
  status\_update & write & Modification d'un etat. \\
 \hline
  status\_del & write & Suppression d'un etat. \\
 \hline
  type & write & Liste des types d�finies et formulaire de nouveau type. \\
 \hline
  type\_add & write & Ajout d'un type. \\
 \hline
  type\_update & write & Modification d'un type. \\
 \hline
  type\_del & write & Suppression d'un type. \\
 \hline
  display & read & Ecran de modification des pr�f�rences d'affichage. \\
 \hline
  dispref\_display & read & Modifie l'affichage d'un �l�ment. \\
 \hline
  dispref\_level & read & Modifie l'ordre d'affichage d'un �l�ment. \\
 \hline
  document\_add & write & Ajout de liens vers des documents. \\
 \hline
\end{tabular}


\subsubsection{Cat�gories d'un contrat}

Cat�gories d'un contrat.\\

\begin{tabular}{|p{3cm}|p{10cm}|}
\hline\textbf{Code cat�gorie} & \textbf{Label cat�gorie} \\
\hline
Priorite & Cat�gorie o� l'on retrouve les priorit�s des contrats qui ont �t� ins�r� dans le module Administration.\\
\hline
Etat & Cat�gorie o� l'on retrouve les �tats des contrats qui ont �t� ins�r� dans le module Administration.\\
\hline
Type & Cat�gorie o� l'on retrouve les types des contrats qui ont �t� ins�r� dans le module Administration.\\
\hline

Contract type & Cat�gorie permettant de choisir entre les contrats de type Client ou Fournisseurs 
Dans la base de donn�es on reserve le champ contract\_kind. Ce champ peut avoir les valeurs
1 (client) ou 2 (fournisseur) ou 0 pour les anciens contrats qui ont �t� non-actualis�.
Dans le formulaire d'inscription d'un nouveau contrat on trouve dans cette zone deux boutons de type radio
avec le nom de la variable radio\_kind. On a utilise la variable contract\_kind pour garder la variable qui sort de la base de donn�e. Si contract\_kind est �gale � 2 alors le bouton avec l'etiquette fournisseur doit �tre selectionn�, et dans le cas contraire le bouton avec l'etiquette client doit �tre selectionn�. Le bouton client est selectionn�  par ''default''.
Dans le module de consultation on trouve cette cat�gorie au-dessus du module avec les informations de la soci�t� sous la forme d'un labelle 'client' ou 'fournisseur'.
\\
\hline
Format & Cat�gorie o� l'on peut choisir entre les contracts de type Coupons ou Dur�e.\\
 Pour la cat�gorie Coupons, le champ avec le nombre de coupons du contract doit �tre renseign�.\\
 Pour la cat�gorie Duree, le champ avec la duree du suport (h) doit �tre renseign�.\\
On considere qu'un coupon est �gal � une heure de support. Le temps du support du module incident est transform�
 en coupons ou en heures de support pour le module contrat.\\
La dur�e disponible du support d'un contrat est �gale � la difference entre la dur�e du support initiale (le nombre 
 d'heures qui ont �t� vendu) et la dur�e utilis�e (le nombre d'heures qui ont �t� necessaires pour resoudre les incidents)
Le nombre de coupons disponibles d'un contrat est �gal � la difference entre le nombre de coupons initial ( qui ont �t� vendus) et la dur�e utilis�e (nombre d'heures qui ont �t� n�cessaires pour r�soudre le(s) incident(s)).\\
Pour gerer le type du contract (par coupons ou par duree) nous avons utilise dans la base de donnee le champ
contract\_ticket , qui peut prendre des valeurs 1 si on a un contrat par coupons ou 2  si on a un contract par duree.\\
Le nombre de coupons et le nombre d'heures de support vendues sont enregistr�s dans la base de donn�es dans les champs:
contract\_ticketnumber et contract\_supportduration.\\
Les deux champs ne peuvent pas �tre rempli en m�me temps avec des valeurs differentes de z�ro.\\
Dans le formulaire d'inscription d'un nouveau contrat on trouve un champ de type select avec trois boutons radios 'P�riode','Duree' et 'Coupons'. Le champ 'P�riode' est selectionn� � l'�tat initial.\\ 
Si le champ 'Coupons' est selectionn� un champ texte permettant la saisie du 'Nombre coupons' appara�t.\\
En mode consultation ces informations apparaissent dans la zone 'etat': nombre de 'coupons' ou heures' consomm�es ou disponibles (en foction des heures n�cessaires pour la r�solution des incidents).\\
S'il n'y a pas d'incidents alors le nombre d'heures ou de coupons consomm�s est �gal � z�ro.\\

\\
\hline
\end{tabular}

% Documentation technique d'OBM : module Facture
% ALIACOM Pierre Baudracco
% $Id$


\clearpage
\section{Facture}

Le module \invoice \obm.

\subsection{Organisation de la base de donn�es}

Le module \invoice utilise 2 tables :
\begin{itemize}
 \item Invoice
 \item InvoiceStatus
\end{itemize}

\subsection{Invoice}
Table principale des informations d'une facture.\\

\begin{tabular}{|p{3cm}|c|p{5.4cm}|p{2.6cm}|}
\hline
\textbf{Champs} & \textbf{Type} & \textbf{Description} & \textbf{Commentaire} \\
\hline
\_id & int 8 & Identifiant & Cl� primaire \\
\hline
\_timeupdate & timestamp 14 & Date de mise � jour & \\
\hline
\_timecreate & timestamp 14 & Date de cr�ation & \\
\hline
\_userupdate & int 8 & Id du modificateur & \\
\hline
\_usercreate & int 8 & Id du createur & \\
\hline
\_company\_id & int 8 & Soci�t� client ou fournisseur & \\
\hline
\_deal\_id & int 8 & Affaire concern�e & \\
\hline
\_project\_id & int 8 & Projet concern� & \\
\hline
\_number & varchar 10 & Num�ro de la facture & \\
\hline
\_label & varchar 128 & Label & \\
\hline
\_amount\_ht & double 10,2 & Montant Hors taxe &\\
\hline
\_amount\_ttc & double 10,2 & Montant TTC &\\
\hline
\_status\_id & int 4 & Etat & \\
\hline
\_date & date & Jour de facturation & pr�vu ou r�el \\
\hline
\_inout & char 1 & Type client ou fournisseur & (+ client, - fournisseur) \\
\hline
\_archive & char 1 & Indicateur d'archivage & (1 = 0ui)\\
\hline
\_comment & text (64k) & Commentaire &\\
\hline
\end{tabular}


\subsection{InvoiceStatus}
Table des informations des �tats des factures.\\

\begin{tabular}{|p{3cm}|c|p{5.4cm}|p{2.6cm}|}
\hline
\textbf{Champs} & \textbf{Type} & \textbf{Description} & \textbf{Commentaire} \\
\hline
\_id & int 8 & Identifiant & Cl� primaire \\
\hline
\_payment & int 1 & Indicateur facture doit avoir des paiements & (1 = Oui) \\
\hline
\_created & int 1 & Indicateur facture cr��e (n�cessite num�ro et dates) & (1 = Oui) \\
\hline
\_archive & char 1 & Indicateur facture de cet Etat est archivable & (1 = Oui)\\
\hline
\_label & varchar 24 & Label de l'�tat & \\
\hline
\end{tabular}


\subsubsection{Valeurs par d�faut}

Il est possible de g�rer ses propres �tats de factures mais des �tats par d�faut sont d�finis :\\

\begin{tabular}{|p{3cm}|c|c|c|p{4cm}|}
\hline
\textbf{Etat} & \textbf{Paiement} & \textbf{Cr�ation} & \textbf{Archivable} & \textbf{Commentaire} \\
\hline
A cr�er & & & & \\
\hline
Envoy�e & X & X & & \\
\hline
Pay�e en partie & X & X & & \\
\hline
Litige & X & X & & \\
\hline
Annul�e & & X & X & \\
\hline
Pertes et profits & & X & X & \\
\hline
Pay�e & X & X & X & \\
\hline
\end{tabular}

\subsubsection{Remarques}

\paragraph{L'ordre d'affichage} des �tats est : ``Non archivable, Pas de paiements'', ce qui donne l'ordre du tableau ci-dessus.


\subsection{ProjectTerm}
Table des conditions de r�glement ou d�lais de paiements.



\subsection{Actions et droits}

Voici la liste des actions du module \invoice, avec le droit d'acc�s requis ainsi qu'une description sommaire de chacune d'entre elles.\\

\begin{tabular}{|l|c|p{9.5cm}|}
 \hline
 \textbf{Intitul�} & \textbf{Droit} & \textbf{Description} \\
 \hline
 \hline
  index & read & (D�faut) formulaire de recherche de factures. \\ 
 \hline
  search & read & R�sultat de recherche de factures. \\
 \hline
  new & write & Formulaire de cr�ation d'une facture. \\
 \hline
  detailconsult & read & Fiche d�tail d'une facture. \\
 \hline
  detailupdate & write & Formulaire de modification d'une facture. \\
 \hline
  duplicate & write & Formulaire de cr�ation � partir d'une facture existante. \\
 \hline
  insert & write & Insertion d'une facture. \\
 \hline
  update & write & Mise � jour d'une facture. \\
 \hline
  check\_delete & write & V�rification avant suppression d'une facture. \\
 \hline
  delete & write & Suppression d'une facture. \\
 \hline
  display & read & Ecran de modification des pr�f�rences d'affichage. \\
 \hline
  dispref\_display & read & Modifie l'affichage d'un �l�ment. \\
 \hline
  dispref\_level & read & Modifie l'ordre d'affichage d'un �l�ment. \\
 \hline
  document\_add & write & Ajout de liens vers des documents. \\
 \hline
\end{tabular}

% Documentation technique d'OBM : module Paiment
% ALIACOM Pierre Baudracco
% $Id$


\clearpage
\section{Paiment}

Le module \payment d'\obm.

\subsection{Organisation de la base de donn�es}

Le module \payment utilise 3 tables :
\begin{itemize}
 \item Payment
 \item PaymentKind
 \item PaymentInvoice
\end{itemize}

\subsection{Payment}
Table principale des informations d'un paiement.\\

\begin{tabular}{|p{3cm}|c|p{5.4cm}|p{2.6cm}|}
\hline
\textbf{Champs} & \textbf{Type} & \textbf{Description} & \textbf{Commentaire} \\
\hline
\_id & int 8 & Identifiant & Cl� primaire \\
\hline
\_timeupdate & timestamp 14 & Date de mise � jour & \\
\hline
\_timecreate & timestamp 14 & Date de cr�ation & \\
\hline
\_userupdate & int 8 & Id du modificateur & \\
\hline
\_usercreate & int 8 & Id du createur & \\
\hline
\_company\_id & int 8 & Soci�t� client ou fournisseur & \\
\hline
\_number & varchar 10 & Num�ro du paiement & \\
\hline
\_date & date & Date du paiment & \\
\hline
\_expected\_date & date & Date pr�vue du paiment & \\
\hline
\_amount & double 10,2 & Montant &\\
\hline
\_label & varchar 128 & Label & \\
\hline
\_paymentkind\_id & int 8 & Moyen de paiement & \\
\hline
\_account\_id & int 8 & Compte cr�dit� ou d�bit� & \\
\hline
\_inout & char 1 & Type entr�e ou sortie & (+ entr�e, - sortie) \\
\hline
\_checked & char 1 & Indicateur de v�rification & (1 = 0ui)\\
\hline
\_comment & text (64k) & Commentaire &\\
\hline
\end{tabular}


\subsection{PaymentKind}
Table des moyens de paiement.\\

\begin{tabular}{|p{3cm}|c|p{5.4cm}|p{2.6cm}|}
\hline
\textbf{Champs} & \textbf{Type} & \textbf{Description} & \textbf{Commentaire} \\
\hline
\_id & int 8 & Identifiant & Cl� primaire \\
\hline
\_shortlabel & char 3 & Label court & \\
\hline
\_longlabel & varchar 40 & Label & \\
\hline
\end{tabular}


\subsection{PaymentInvoice}
Table de liaison entre les paiements et les factures.\\
Une facture peut avoir plusieurs paiement et un paiement peut couvrir plusieurs factures.

\begin{tabular}{|p{3cm}|c|p{5.4cm}|p{2.6cm}|}
\hline
\textbf{Champs} & \textbf{Type} & \textbf{Description} & \textbf{Commentaire} \\
\hline
\_invoice\_id & int 8 & Identifiant de la facture & \\
\hline
\_payment\_id & int 8 & Identifiant du paiment & \\
\hline
\_amount & double 10,2 & Montant concern� du paiment pour la facture &\\
\hline
\_timeupdate & timestamp 14 & Date de mise � jour & \\
\hline
\_timecreate & timestamp 14 & Date de cr�ation & \\
\hline
\_userupdate & int 8 & Id du modificateur & \\
\hline
\_usercreate & int 8 & Id du createur & \\
\hline
\end{tabular}



\subsection{Actions et droits}

Voici la liste des actions du module \payment, avec le droit d'acc�s requis ainsi qu'une description sommaire de chacune d'entre elles.\\

\begin{tabular}{|l|c|p{9.5cm}|}
 \hline
 \textbf{Intitul�} & \textbf{Droit} & \textbf{Description} \\
 \hline
 \hline
  index & read & (D�faut) formulaire de recherche des paiments. \\ 
 \hline
  search & read & R�sultat de recherche de paiments. \\
 \hline
  new & write & Formulaire de cr�ation d'un paiment. \\
 \hline
  detailconsult & read & Fiche d�tail d'un paiment. \\
 \hline
  detailupdate & write & Formulaire de modification d'un paiment. \\
 \hline
  duplicate & write & Formulaire de cr�ation � partir d'une facture existante. \\
 \hline
  insert & write & Insertion d'un paiment. \\
 \hline
  update & write & Mise � jour d'un paiment. \\
 \hline
  check\_delete & write & V�rification avant suppression d'un paiment. \\
 \hline
  delete & write & Suppression d'un paiment. \\
 \hline
  display & read & Ecran de modification des pr�f�rences d'affichage. \\
 \hline
  dispref\_display & read & Modifie l'affichage d'un �l�ment. \\
 \hline
  dispref\_level & read & Modifie l'ordre d'affichage d'un �l�ment. \\
 \hline
\end{tabular}

% Documentation technique d'OBM : module references
% ALIACOM Pierre Baudracco
% $Id$


\clearpage
\section{Référentiel (module \adminref)}

Le module \adminref d'\obm.

\subsection{Organisation de la base de données}

Le module \adminref permet de gérer 3 tables de références :
\begin{itemize}
 \item TaskType
 \item Country
 \item DataSource
\end{itemize}

\subsection{TaskType}

Table des types de tâche, utilisée par les modules :
\begin{itemize}
 \item \deal
 \item \project
 \item \timemanager
\end{itemize}
\vspace{0.4cm}

\begin{tabular}{|p{3cm}|c|p{5.4cm}|p{2.6cm}|}
\hline
\textbf{Champs} & \textbf{Type} & \textbf{Description} & \textbf{Commentaire} \\
\hline
\_id & int 8 & Identifiant & Clé primaire \\
\hline
\_timeupdate & timestamp 14 & Date de mise à jour & \\
\hline
\_timecreate & timestamp 14 & Date de création & \\
\hline
\_userupdate & int 8 & Id du modificateur & \\
\hline
\_usercreate & int 8 & Id du créateur & \\
\hline
\_label & varchar 255 & Label du type de tâche & \\
\hline
\_internal & int 1 & Catégorie du type de tâche & \\
\hline
\end{tabular}


\subsubsection{Catégories de TaskType}

\begin{tabular}{|p{3cm}|c|p{5.4cm}|}
\hline
\textbf{Variable} & \textbf{Valeur} & \textbf{Description} \\
\hline
\$ctt\_sales & 0 & Tâches de production \\
\hline
\$ctt\_research & 1 & Tâches de R\&D interne \\
\hline
\$ctt\_others & 2 & Tâches non liées à la production \\
\hline
\end{tabular}


\subsection{Country}
Table de référence des pays.\\

Table de référence des pays, utilisée par les modules :
\begin{itemize}
 \item \company
 \item \contact
 \item \List
 \item \import
\end{itemize}
\vspace{0.4cm}

\begin{tabular}{|p{3cm}|c|p{5.4cm}|p{2.6cm}|}
\hline
\textbf{Champs} & \textbf{Type} & \textbf{Description} & \textbf{Commentaire} \\
\hline
\_timeupdate & timestamp 14 & Date de mise à jour & \\
\hline
\_timecreate & timestamp 14 & Date de création & \\
\hline
\_userupdate & int 8 & Id du modificateur & \\
\hline
\_usercreate & int 8 & Id du createur & \\
\hline
\_iso3166 & char 2 & Code ISO 3166 du pays & (FR, IT,..) \\
\hline
\_name & varchar 64 & Nom du pays dans la langue indiquée & \\
\hline
\_lang & char 2 & Langue du nom du pays & \\
\hline
\_phone & varchar 5 & Indicateur téléphonique du pays & \\
\hline
\end{tabular}

\subsubsection{Remarques}

L'identifiant d'un pays est le code ISO 3166. Mais ce n'est pas la clé primaire car un pays peut être présent plusieurs fois. Chaque entrée propose le nom du pays dans une seule langue (référencée par le champ country\_lang).

la clé primaire est le couple (Code ISO 3166, LANG).


\subsection{DataSource}

Table de référence des sources de données, utilisée par les modules :
\begin{itemize}
 \item \company
 \item \contact
 \item \import
\end{itemize}
\vspace{0.4cm}

\begin{tabular}{|p{3cm}|c|p{5.4cm}|p{2.6cm}|}
\hline
\textbf{Champs} & \textbf{Type} & \textbf{Description} & \textbf{Commentaire} \\
\hline
\_id & int 8 & Identifiant & Clé primaire \\
\hline
\_timeupdate & timestamp 14 & Date de mise à jour & \\
\hline
\_timecreate & timestamp 14 & Date de création & \\
\hline
\_userupdate & int 8 & Id du modificateur & \\
\hline
\_usercreate & int 8 & Id du createur & \\
\hline
\_name & varchar 64 & Nom de la source de données & \\
\hline
\end{tabular}


\subsection{Actions et droits}

Voici la liste des actions du module \project, avec le droit d'accès requis ainsi qu'une description sommaire de chacune d'entre elles.\\

\begin{tabular}{|l|c|p{9.5cm}|}
 \hline
 \textbf{Intitulé} & \textbf{Droit} & \textbf{Description} \\
 \hline
 \hline
  country & read\_admin & Ecran de gestion des pays.\\ 
 \hline
  country\_insert & write\_admin & Ajout d'un pays. \\
 \hline
  country\_update & write\_admin & Modification d'un pays. \\
 \hline
  country\_checklink & write\_admin & Vérification avant suppression d'un pays. \\
 \hline
  country\_delete & write\_admin & Suppression d'un pays. \\
 \hline
  datasource & read\_admin & Ecran de gestion des sources de données.\\ 
 \hline
  datasource\_insert & write\_admin & Ajout d'une source de données. \\
 \hline
  datasource\_update & write\_admin & Modification d'une source de données. \\
 \hline
  datasource\_checklink & write\_admin & Vérification avant suppression d'une source de données. \\
 \hline
  datasource\_delete & write\_admin & Suppression d'une source de données. \\
 \hline
  tasktype & read\_admin & Ecran de gestion des types de tâche.\\ 
 \hline
  tasktype\_insert & write\_admin & Ajout d'un type de tâche. \\
 \hline
  tasktype\_update & write\_admin & Modification d'un type de tâche. \\
 \hline
  tasktype\_checklink & write\_admin & Vérification avant suppression d'un type de tâche. \\
 \hline
  tasktype\_delete & write\_admin & Suppression d'un type de tâche. \\
 \hline
\end{tabular}

% Documentation technique d'OBM : module Import
% ALIACOM Pierre Baudracco
% $Id$


\clearpage
\section{Import}

Le module \import \obm.

\subsection{Organisation de la base de données}

Le module \import utilise 1 table :
\begin{itemize}
 \item Import
\end{itemize}

\subsubsection{La table Import}
Table principale des informations d'un import.\\

\begin{tabular}{|p{3cm}|c|p{5.4cm}|p{2.6cm}|}
\hline
\textbf{Champs} & \textbf{Type} & \textbf{Description} & \textbf{Commentaire} \\
\hline
\_id & int 8 & Identifiant & Clé primaire \\
\hline
\_timeupdate & timestamp 14 & Date de mise à jour & \\
\hline
\_timecreate & timestamp 14 & Date de création & \\
\hline
\_userupdate & int 8 & Id du modificateur & \\
\hline
\_usercreate & int 8 & Id du créateur & \\
\hline
\_name & varchar 64 & Nom de l'import & \\
\hline
\_datasource\_id & int 8 & Source de données & \\
\hline
\_marketingmanager\_id & int 8 & Responsable affecté aux données & \\
\hline
\_separator & char 3 & Séparateur de champ & \\
\hline
\_enclosed & char 1 & caractère d'encadrement des champs & \\
\hline
\_desc & text (64k) & Description de l'import (mapping des champs,...) &\\
\hline
\end{tabular}


\subsubsection{Le champ description (mapping des champs)}

L'idée globale du fonctionnement de l'import est d'effectuer un mapping des champs du fichier à importer dans les champs définis de la base de données.

Ce mappin gest stocké dans le champ \variable{import\_desc}, sous forme de définitions de variables, qui est évalué par OBM.\\

Les informations concernant les champs société sont stockées dans le tableau global \variable{\$comp}, celles concernant les champs contact dans \variable{\$con}.

Pour chaque champ défini et proposé par le module \import, le champ description stocke 3 informations : Exemple pour le nom de la société, le tableau \variable{\$comp[``comp\_name'']} contient ces 3 informations :\\

\begin{tabular}{|p{3cm}|p{5.6cm}|p{2.6cm}|}
\hline
\textbf{Champs} & \textbf{Description} & \textbf{Exemple} \\
\hline
[``value''] & numéro de colonne dans fichier & 1 \\
\hline
[``label''] & nom du label du champ & l\_company \\
\hline
[``default''] & valeur par défaut si non présent & ZZ \\
\hline
\end{tabular}


\subsection{Précisions sur le mapping des champs}

L'association des champs du fichier aux champs de la base de données utilise plusieurs structures :\\

\begin{tabular}{|p{1.5cm}|p{3.6cm}|p{4.8cm}|p{4cm}|}
\hline
\textbf{Structure} & \textbf{Description} & \textbf{Exemple} & \textbf{Fonction}\\
\hline
\$row[] & Tableau des valeurs d'une ligne du fichier & \$row[0]=Aliacom & \fonction{fgetcsv de PHP} \\
\hline
\$map[] & Mapping descriptif de l'import & \$map[l][field]=comp\_name \$map[1][label]=l\_company \$map[default][comp\_name]=ZZ \$map[field][comp\_name]=l & \fonction{get\_import\_field\_mapping}\\
\hline
\$fields[] & Valeurs des champs & \$fields[comp\_name]=Aliacom & \fonction{get\_import\_row\_mapping}\\
\hline
\end{tabular}

\vspace{0.3cm}
La structure \variable{\$row} est récupérée par une lecture de la ligne du fichier en tenant compte des paramètres CSV (séparateur, caractère encadrant).
L'utilisation de la fonction \fonction{fgetcsv()} de PHP permet de gérer les caractères encadrant optionnels (champs encadrés optionnellement par ").\\

La structure \variable{\$map} est construite à partir du mapping entré par l'utilisateur.
Elle comporte 3 sous structures :\\

\begin{tabular}{|p{2cm}|p{3.2cm}|p{4.8cm}|p{3.8cm}|}
\hline
\textbf{Structure} & \textbf{Description} & \textbf{Exemple} & \textbf{Commentaire}\\
\hline
\$map[N] & Mapping num colonne vers champ & \$map[l][field]=comp\_name \$map[1][label]=l\_company & N numéro de colonne du fichier. Utile pour afficher exemple du fichier \\
\hline
\$map[default] & Mapping valeurs des champs par défaut & \$map[default][comp\_name]=ZZ & \\
\hline
\$map[field] & Mapping champ recoit colonne & \$map[field][comp\_name]=l & Permet qu'une colonne soit mappée sur plusieurs champs\\
\hline
\end{tabular}

\vspace{0.3cm}
La structure \variable{\$fields} est construite à partir des 2 structures précédents (\variable{\$row} et \variable{\$map}. Elle contient le mapping final des valeurs dans les champs.


\subsection{Actions et droits}

Voici la liste des actions du module \import, avec le droit d'accès requis ainsi qu'une description sommaire de chacune d'entre elles.\\

\begin{tabular}{|l|c|p{9.5cm}|}
 \hline
 \textbf{Intitulé} & \textbf{Droit} & \textbf{Description} \\
 \hline
 \hline
  index & read\_admin & (Défaut) formulaire de recherche d'imports. \\ 
 \hline
  search & read\_admin & Résultat de recherche. \\
 \hline
  new & write\_admin & Formulaire de création d'un import. \\
 \hline
  detailconsult & read\_admin & Fiche détail d'un import. \\
 \hline
  detailupdate & write\_admin & Formulaire de modification d'un import. \\
 \hline
  insert & write\_admin & Insertion d'un import. \\
 \hline
  update & write\_admin & Mise à jour d'un import. \\
 \hline
  check\_delete & write\_admin & Vérification avant suppression d'un import. \\
 \hline
  delete & write\_admin & Suppression d'un import. \\
 \hline
  file\_sample & write\_admin & Vérification d'un fichier. Affichage 1eres lignes\\
 \hline
  file\_test & write\_admin & Test d'un import du fichier. \\
 \hline
  file\_import & write & Importer un fichier. \\
 \hline
\end{tabular}

\clearpage
\section{Etendre OBM}
% Documentation technique d'OBM : Etendre OBM : Ajouter un champ
% ALIACOM Pierre Baudracco
% $Id$


%%\clearpage
\subsection{Ajouter un champ dans un module}

Différentes étapes sont nécessaires pour ajouter un champ dans un module. Selon l'utilisation du nouveau champ, certaines étapes peuvent êtres ignorées (ex: inclusion dans moteur de recherche, affichage dans résultat,...).\\

\begin{itemize}
 \item Modifier les scripts de création de base de données pour inclure le nouveau champ.
   \begin{verbatim}
emacs scripts/0.8/create_obmdb_0.8.mysql.sql
emacs scripts/0.8/create_obmdb_0.8.pgsql.sql
   \end{verbatim}
 \item Modifier (ou créer) les scripts de mise à jour de base de données pour inclure le nouveau champ.
   \begin{verbatim}
emacs scripts/0.8/update_0.8.1-0.8.2.mysql.sql
emacs scripts/0.8/update_0.8.1-0.8.2.pgsql.sql
   \end{verbatim}
 \item (module\_index.php) Les paramètres sont récoltés automatiquement. Mais si le nouveau paramètre doit subir un traitement spécial, ceci doit être implémenté dans (\fonction{get\_module\_params()})\\
 \item Si champ sélection d'une liste déroulante :
    \begin{itemize}
     \item[*] Créer la table associée (scripts création et mise à jour bases de données)\\
     \item[*] (module\_index.php) Appeler la liste de valeurs run\_query\_fieldname() pour la passer aux fonctions formulaire (mise à jour voire recherche).\\
     \item[*] (module\_display.inc) Modifier la définition des fonctions formulaire afin d'inclure la liste de valeur (html\_module\_form(), html\_module\_search\_form()).\\
     \item[*] (module\_query.inc) Implémenter les fonctions run\_query\_fieldname(), run\_query\_fieldname\_insert(), run\_query\_fieldname\_update(), run\_query\_fieldname\_delete(), run\_query\_fieldname\_links().
      \begin{verbatim}
function run_query_companynafcode()
function run_query_nafcode_insert()
function run_query_nafcode_update()
function run_query_nafcode_delete()
function run_query_nafcode_links()
      \end{verbatim}
     \item[*] (module\_display.inc) Implémenter les fonctions de gestion et visualisation : html\_module\_field\_form(), dis\_field\_links().
      \begin{verbatim}
function html_company_nafcode_form()
function dis_nafcode_links()
      \end{verbatim}
     \item[*] (module\_index.php) Définir les actions avec les droits associés (get\_module\_action()) et les implémenter dans le branchement global du module (if \$action ==...) avec la traduction des messages (insert\_ok, insert\_error,...).\\
     \item[*] (module\_js.inc) Implémenter les actions Javascript de vérification des formulaire de gestion (check\_field\_upd(), check\_field\_new(), check\_field\_checkdel()).\\
    \end{itemize}
 \item (lang/*/module.inc) Créer les variables de langues (au moins pour le français et l'anglais).\\
 \item Si le champ peut être masqué, indiquer l'option dans obminclude/global.inc et obm\_conf.inc et mettre la doc à jour (manuel\_obm/technique/t\_site.tex).\\
 \item (module\_display.inc) Prendre en compte le nouveau champ dans les fonctions d'affichage (html\_module\_form(), html\_module\_consult(), dis\_module\_warn\_insert()).\\
 \item Si le champ peut être recherché l'implémenter dans les fonctions html\_module\_search\_form(), html\_module\_search\_list() pour propagation du résultat, et dans la fonction de recherche run\_query\_search().\\
 \item (module\_query.inc) Prendre en compte le nouveau champ dans les fonctions base de données : run\_query\_detail(), run\_query\_insert(), run\_query\_update(), dis\_module\_warn\_insert()).\\
 \item Si le champ doit figuré dans les résultats de recherche, il faut l'ajouter au module correspondant dans les préférences par défaut (scripts/0.8/obmdb\_default\_values\_0.8.sql).
\end{itemize}


%\input{user/u_manuelr.tex}

\end{document}
