\section{Installation pas à pas d'OBM 1.1.x sous Windows}
 

Cette documentation est valide pour la version 1.1.0. Seules les
manipulations spécifiques à OBM sont décrites. Pour l'installation de
PHP, MySQL, PostgreSQL, Apache ou autres composants libres, des exemples sont proposés correspondant à la version Windows XP utilsé ave easyPHP mais il est préférable de vous référer à la documentation de votre distribution.


\subsection{Hypothèses}

Cette documentation suppose que vous disposez d'un serveur Apache,
d'un serveur de BD MySQL et de PHP correctement configurés (Apache et mysql et php sont directement inclus dans easyPHP).

Dans la suite du document, on suppose que la racine d'OBM se trouve
dans \fichier{c:$\backslash$obm} et qu'easy-php se trouve dans \fichier{c:\textbackslash program files\textbackslash easyPHP$\backslash$ }
\fichier{note\,:} si le port 80 est occupé et empêche Apache de se lancer, ouvrez un explorateur, 
allez dans le répertoire d'installation d'easyPHP, puis éditez le fichier httpd.conf du répertoire conf\_files.

Recherchez la ligne où est inscrit Port 80 et remplacez la par
\shadowbox{
  \begin{minipage}{13cm}
\begin{verbatim}
Port 81
\end{verbatim}
  \end{minipage}
}

\subsection{Télécharger les sources}

L'installation serait plus difficile sans...

Il faut donc télécharger les sources et les décompresser dans
\fichier{c:$\backslash$obm} (en fait dans le répertoire qui contiendra la
racine de votre OBM). 
Vous pouvez télécharger les sources à l'addresse suivante :
\begin{verbatim}
http://obm.aliacom.fr/download.php
\end{verbatim}
Pour décompresser l'archive obm-1.1.0.tar.gz, 
il vous faudra certainement un logiciel de décompression,
 winrar (que vous pouvez télécharger sur www.telecharger.com) ou un autre 
fera parfaitement l'affaire.
Copiez ensuite les répertoires \fichier{php} et \fichier{obminclude} dans le repertoire \fichier{www} d'easyPHP


\subsection{Configuration d'Apache et de PHP}

\subsubsection*{virtual host}
Nous allons  configurer un \emph{virtual host} pour gérer une instance
d'OBM (vous pouvez avoir plusieurs \emph{virtuals hosts} sur un seul
serveur). 

Dans la section \emph{virtual host} du \fichier{httpd.conf},
positionner le \emph{Document Root} à C:$\backslash$obm$\backslash$php. Vous pouvez
accéder à ce fichier en cliquant sur le cadre en haut à gauche de la fenêtre
d'easy-php puis sélectionnez configuration et Apache.
Le virtual Host doit ressembler à ce qui suit. Les fichier de la fin concernent les logs d'erreurs et d'accès à OBM.
L'ensemble des addresse dépend aussi de la configuration que vous aurez choisi.

\shadowbox{
  \begin{minipage}{13cm}
\begin{verbatim}
NameVirtualHost <nom_du_virtualhost obm par exemple>

    <VirtualHost VotreIpIci>
       ServerAdmin root@localhost
       DocumentRoot "c:\obm\php"
       ServerName obm
       Alias /images "c:\obm\obminclude\themes"
       DirectoryIndex obm.php
       ErrorLog "${path}/Apache/log/obm-error.log"
       CustomLog "${path}/Apache/log/obm-access.log" common
    </VirtualHost>
\end{verbatim}
  \end{minipage}
}

dans ce même fichier, la ligne suivante doit être présente :

\shadowbox{
  \begin{minipage}{13cm}
\begin{verbatim}
	AddType application/x-httpd-php .php
\end{verbatim}
  \end{minipage}
}


%\subsubsection*{Répertoires d'include d'OBM}
%
%Le nom du répertoire d'include (\fichier{obminclude} par défaut) est
%maintenant une variable d'environnement pour permettre à plusieurs
%instances d'OBM basées sur le même répertoire de source de tourner
%simultanément. Seul ce répertoire est à modifier sur chaque instance
%car il contient les réglages spécifiques (thèmes, bases de données,
%langues) pour chaque instance.
%
%Cette variable est donc positionnée dans le fichier
%\fichier{httpd.conf} d'Apache. Il faut d'abord charger le module
%\fichier{env} : 

%\shadowbox{
%  \begin{minipage}{13cm}
%\begin{verbatim}
%LoadModule env_module /usr/lib/apache/1.3/mod_env.so (apache 1.3 on Debian)
%LoadModule env_module modules/mod_env.so (apache 2 on Redhat)
%\end{verbatim}
%  \end{minipage}
%}
%
%Puis renseigner la variable OBM\_INCLUDE\_VAR avec le nom du
%répertoire d'include : 

%\shadowbox{
%  \begin{minipage}{13cm}
%\begin{verbatim}
%Setenv OBM_INCLUDE_VAR obminclude
%\end{verbatim}
%  \end{minipage}
%}

%Le chemin d'accès au répertoire d'include d'OBM doit être donné
%(D'ailleurs, le répertoire \fichier{obminclude} peut être déplacé) :  
%
%ùRenseigner la variable include\_path avec le chemin vers le répertoire
%d'include : 
%
%\shadowbox{
%  \begin{minipage}{13cm}
%\begin{verbatim}
%php_value include_path ".;c:\obm"
%\end{verbatim}
%  \end{minipage}
%}


%\subsubsection*{Alias pour les images}
%
%Il faut installer un alias images vers le répertoire de themes : 
%Cela se fait en bas du fichier httpd.conf
%\shadowbox{
%  \begin{minipage}{13cm}
%\begin{verbatim}
%Alias /images c:\obm\obminclude\themes
%\end{verbatim}
%  \end{minipage}
%}

%\subsubsection*{Directory Index}
%
%Il faut enfin spécifier le fichier par défaut : 
%
%\shadowbox{
%  \begin{minipage}{13cm}
%\begin{verbatim}
%DirectoryIndex obm.php
%\end{verbatim}
%  \end{minipage}
%}

%Nous recommandons \textbf{chaudement} d'interdire l'accès direct aux fichiers
%\fichier{.inc} :

%\shadowbox{
%  \begin{minipage}{13cm}
%\begin{verbatim}
%<Files ~ "\.inc$">
%   Order allow,deny
%   Deny from all
%</Files>
%\end{verbatim}
%  \end{minipage}
%}


%\subsubsection*{Section virtual host d'exemple complète}
%
%Vérifier que votre IP soit définie comme un \emph{named virtual
%  host} et insérer cette section :
%
%\shadowbox{
%  \begin{minipage}{13cm}
%\begin{verbatim}
%NameVirtualHost 192.168.1.5
%
%<VirtualHost 192.168.1.5>
%       ServerAdmin root@localhost
%       DocumentRoot "c:\obm\php"
%       ServerName obm
%       Alias /images "c:\obm\obminclude\themes"
%       DirectoryIndex obm.php
%    </VirtualHost> 
%\end{verbatim}
%  \end{minipage}
%}

\subsubsection*{configuration de php}

Dans le menu d'easyPHP aller dans configurer -> PHP, ou bien editez le fichier \fichier{php.ini} dans le répertoire \fichier{cong\_files} d'easyPHP.

remplacez la ligne : 

\shadowbox{
  \begin{minipage}{13cm}
\begin{verbatim}
include_path = ".;c:\EasyPhp\php\pear\"
\end{verbatim}
  \end{minipage}
}

par la ligne 

\shadowbox{
  \begin{minipage}{13cm}
\begin{verbatim}
include_path = ".;c:\EasyPhp\php\pear\;c:\obm\"
\end{verbatim}
  \end{minipage}
}

\shadowbox{
  \begin{minipage}{13cm}
\begin{verbatim}
error_reporting = E_ALL 
\end{verbatim}
  \end{minipage}
}

par la ligne 

\shadowbox{
  \begin{minipage}{13cm}
\begin{verbatim}
error_reporting = E_ALL & ~E_NOTICE 
\end{verbatim}
  \end{minipage}
}

\fichier{Attention, on suppose toujours OBM enstallé dans c:$\backslash$obm}\\
\\
Vérifier les configurations de register\_global, safe\_mode, magic\_quotes\_gpc
\shadowbox{
  \begin{minipage}{13cm}
\begin{verbatim}
register_global = On
safe_mode = Off
magic_quotes_gpc = On
allow_call_time_pass_reference = On
\end{verbatim}
  \end{minipage}
}

La ligne suivante doit être présente, dans la partie extension.

\shadowbox{
  \begin{minipage}{13cm}
\begin{verbatim}
mysql.dll
\end{verbatim}
  \end{minipage}
}
\subsection{Configuration d'\obm}

La configuration d'\obm se trouve dans le fichier \fichier{obminclude$\backslash$obm\_conf.inc}.

Pour le créer, copier le fichier \fichier{obminclude$\backslash$obm\_conf.inc.sample} 
dans le nouveau fichier \fichier{obminclude$\backslash$obm\_conf.inc}, et éditer ce dernier.


\subsubsection{Configuration initiale pour la base de données OBM}

Éditer le fichier \fichier{obm\_conf.inc} dans \fichier{obminclude} et

\begin{itemize}
\item choisir la base de données à utiliser ; 
\item la déclarer ;
\item déclarer le nom d'utilisateur et le mot de passe à utiliser.
\item déclarer l'url d'accès à OBM (\$cgp\_host).
\item déclarer le dépôt de documents (voir \ref{install_doc}).
\end{itemize}

Il est également possible de modifier des préférences globales dans ce
fichier, comme cgp\_mail\_enabled par exemple.

\shadowbox{
  \begin{minipage}{13cm}
\begin{verbatim}
// Database infos
$obmdb_host = "localhost";
$obmdb_dbtype = "MYSQL"; // (MYSQL || PGSQL)
$obmdb_db = "obm";
$obmdb_user = "obm";
$obmdb_password = "obm";
$cgp_host = "http://obm/";
...
$cdocument_root = "C:\obm\documents\";
...
// is Mail enabled ? (agenda)
$cgp_mail_enabled = false;
\end{verbatim}
  \end{minipage}
}


\subsubsection*{Configurer le dépôt de documents}
\label{install_doc}

Éditer le fichier \fichier{obm\_conf.inc} dans \fichier{obminclude}.

\shadowbox{
  \begin{minipage}{13cm}
\begin{verbatim}
$cdocument_root = "c:\obm\obmdocuments";
\end{verbatim}
  \end{minipage}
}


%Créer le répertoire et le faire appartenir à l'utilisateur exécutant le serveur apache (www-data sous Debian).

%\shadowbox{
%  \begin{minipage}{13cm}
%\begin{verbatim}
%mkdir /var/www/obmdocuments
%chown www-data:www-data /var/www/obmdocuments
%\end{verbatim}
%  \end{minipage}
%}


\subsubsection*{Configuration et création de la base de données}

Il faut d'abord copier les fichiers de création de la base de donnée dans le répertoire bin de MYSQL. 
Pour le faire, ouvrez un explorateur et allez dans \fichier{C:$\backslash$obm$\backslash$scripts$\backslash$1.1$\backslash$}
Copiez les fichiers \fichier{create\_obmdb\_1.1.mysql.sql},\fichier{obmdb\_test\_values\_1.1.sql},
\fichier{obmdb\_default\_values\_1.1.sql} dans \fichier{c:$\backslash$program files$\backslash$easyPHP$\backslash$mysql$\backslash$bin}
Ensuite, copiez les deux fichiers présents dans fr ou en selon la langue que vous préférez vers le même dossier.

%\subsubsection*{Langue par défaut des données}
%
%Certaines valeurs de références sont insérées par défaut. Ces valeurs
%peuvent être en français (par défaut) ou en anglais. Pour passer de
%l'un à l'autre, il faut éditer le fichier
%\fichier{scripts/1.1/install\_obmdb\_1.1.mysql.sh} (ou
%\fichier{scripts/1.1/install\_obmdb\_1.1.pgsql.sh}) pour changer la
%valeur de DATA\_LANG à \texttt{en} : 
%
%\shadowbox{
%  \begin{minipage}{13cm}
%\begin{verbatim}
%# Mysql User, Password and Data lang var definition 
%U=obm 
%P="obm" 
%DB="obm" 
%DATA_LANG="en"
%\end{verbatim}
%  \end{minipage}
%}

Ensuite, il faut créer la base de données MySQL :
lancer le menu démarer, choisir ligne de commande dans programmes ou choisir éxécuter puis taper sur entrée.

Vous devez voir le prompt suivant.
\shadowbox{
	\begin{minipage}{13cm}
		\begin{verbatim}
			C:\<Documents and Settinfs>\<noms_de_session>
		\end{verbatim}
	\end{minipage}
}

Revenir à la racine et aller dans le répertoire bin de MySQL : 
\begin{verbatim}
cd ..$\backslash$..
cd program files$\backslash$easyPHP$\backslash$MySQL$\backslash$bin			
\end{verbatim}

Se connecter à MySQL et créer la base de données:

\shadowbox{
	\begin{minipage}{13cm}
		\begin{verbatim}
			mysql -u root
			CREATE DATABASE obm;
		\end{verbatim}
	\end{minipage}
}

Remplir la base de données : 
	
\shadowbox{
	\begin{minipage}{13cm}
		\begin{verbatim}
			use obm;
			\. create_obmdb_1.1.mysql.sql
			\.  obmdb_ref_1.1.sql
			\. obmdb_nafcode_1.1.sql
			\. obmdb_test_values_1.1.sql
			\. obmdb_default_values_1.1.sql
		\end{verbatim}
	\end{minipage}
}

Il faut ensuite créer un compte utilisateur. On a ici pris l'exemble où la table s'appelle obm, l'utilisateur est obm@localhost 
et le mot de passe obm. Il faut prendre les mêmes noms que ceux présents dans le fichier \fichier{obm\_conf.inc}

\shadowbox{
  \begin{minipage}{13cm}
\begin{verbatim}
GRANT ALL ON obm.* TO obm@localhost IDENTIFIED BY 'obm';
\end{verbatim}
  \end{minipage}
}



%\subsubsection*{Configuration d'OBM}
%
%Éditer le fichier \fichier{obm\_conf.inc} dans \fichier{obminclude} et
%
%\begin{itemize}
%\item spécifier le mode d'authentification "CAS"; 
%\item déclarer les paramètres de connection au serveur CAS.
%\end{itemize}
%
%Les variables suivantes du fichier de configuration doivent être rennseignées:
%
%\shadowbox{
%  \begin{minipage}{13cm}
%\begin{verbatim}
%
%...
%// authentification : 'CAS' (SSO) or 'standalone' (default)
%$auth_kind="CAS";
%$cas_server = "sso.aliacom.local";
%$cas_server_port = "8443";
%$cas_server_uri = "/cas";
%...
%
%\end{verbatim}
%  \end{minipage}
%}

\subsection{Lancer OBM (accéder à obm.php depuis un navigateur)}

D'abord, redémarrer le serveur web apache à partir d'easyPHP.

Si tout marche bien, lancer un navigateur (Firefox par exemple) et
aller à l'URL : \url{http://yourvirtualhost/}, puis se connecter en
utilisant le compte \texttt{uadmin/padmin}.


Attention : la création de documents et de backups ne marche pas.
