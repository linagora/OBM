% Documentation technique d'OBM : Gestion des affichages "dataset"
% ALIACOM Pierre Baudracco
% $Id$

\subsection{Gestion des affichages des listes de données ou dataset}

Les listes de données ou dataset sont les objets représentant les ensembles de données comme par exemple les résultats d'une recherche.
Ils permettent d'afficher tout résultat de recherche ou requète en proposant les fonctionnalités suivantes :\\

\begin{itemize}
\item Gestion de la pagination
\item Personnalisation par utilisateur des colonnes affichées (voir \ref{display_prefs})
\item Gestion des tris selon toutes les colonnes
\item Export CSV de la page courante ou de l'ensemble du résultat
\item Gestion de case à cocher pour sélection
\end{itemize}
\vspace{0.3cm}
Ces dataset sont gérés par la classe \variable{OBM\_DISPLAY} fournie par le fichier \fichier{of/of\_display.inc}


\subsubsection{Classe of\_display}

\begin{longtable}{|p{3cm}|c|p{8cm}|}
\hline
\textbf{Attributs} & \textbf{Type, valeur} & \textbf{Description} \\
\hline
display\_type & DATA | PREFERENCES & Type de représentation
\begin{itemize}
\item DATA pour les ensembles de données
\item PREFERENCES pour les écrans de préférences
\end{itemize}\\
\hline
display\_name & string & Nom du dataset (nécessaire si plusieurs dataset identiques dans le même écran)\\
\hline
display\_module & string & Module gérant l'entité (ex: deal pour entité parendeal) \\
\hline
display\_entity & string & Entité à afficher (company, contact, list\_contact,..)\\
\hline
display\_pref & prefs array & hash retourné par la fonction \fonction{get\_display\_pref()}\\
\hline
display\_link & \textbf{true} | false & Indique si les liens doivent être affichés (ex: non dans une fenêtre popup)\\
\hline
display\_ext & string & Indique l'action externe appelée dans le contexte d'action externe (ex: get\_id, get\_id\_url) qui permet un affichage spécial des liens par exemple\\
\hline
data\_set & DB\_OBM & Ensemble de données à afficher\\
\hline
data\_header & \textbf{top} | both & Affichage de la line de titre des colonnes\\
\hline
data\_page & \textbf{true} | false & Activation de la gestion de la pagination\\
\hline
data\_url & string & Base de url de recherche (critères, pagination et ordre)
ex: contact\_index.php?action=search\&tf\_lname=baud\\
\hline
data\_order & \textbf{true} | false & Activation des tris par colonne\\
\hline
data\_cb\_text & string & Label de la colonne contenant les checkbox\\
\hline
data\_cb\_show & \textbf{1} | string & Indicateur des lignes pour affichage des checkbox
\begin{itemize}
\item 1 : sur chaque ligne
\item string : nom de colonne définissant si la checkbox doit être affichée (colonne à vrai, checkbox affichée sinon non
\end{itemize}
\\
\hline
data\_cb\_side & \textbf{left} | right & Position de la colonne des checkbox\\
\hline
data\_cb\_name & string & Préfixe du nom des checkbox. Est complété par l'id de la ligne\\
\hline
data\_cb\_field & string & Nom du champ servant à pré-remplir les checkbox. Si le champ est évalué à vrai alors checkbox cochée\\
\hline
data\_idfield & string & Nom du champ servant d'index pour la ligne\\
\hline
\end{longtable}


\begin{longtable}{|p{7cm}|p{8cm}|}
\hline
\textbf{Méthodes publiques} & \textbf{Description} \\
\hline
OBM\_DISPLAY(\$type, \$pref, \$module='''', \$entity='''', \$name=''''
&
Constructeur
\begin{itemize}
\item \$type (display\_type)
\item \$pref (display\_pref)
\item \$module (display\_module)
\item \$entity (display\_type) [optionnel] si non positionné mis par défaut à \$module
\item \$name (display\_name) [optionnel] permet de différencier 2 dataset de la même entité sur le même écran (ex: liste de contact statique et dynamique d'une liste)
\end{itemize}\\
\hline
display(\$module\_function) & Affiche l'ensemble de données
\begin{itemize}
\item \$module\_function : nom de la fonction d'affichage spécifique au module
\end{itemize}\\
\hline
dis\_pref\_help() & Affichage de l'aide de l'écran de préférences \\
\hline
dis\_data\_file(\$first\_row, \$nb\_rows, \$sep, \$dis\_data\_entity) & Affiche les données au format CSV
\begin{itemize}
\item \$first\_row : première ligne à afficher dans l'ensemble des données
\item \$nb\_rows : nombre de lignes à afficher
\item \$sep séparateur de colonnes utilisé
\item \$dis\_data\_entity : fonction gérant l'affichage spécifique de certaines colonnes
\end{itemize}
\\
\hline
\end{longtable}



\subsubsection{Implémentation : API fonctions publiques}
