% Documentation technique d'OBM : module Groupe
% ALIASOURCE Pierre Baudracco
% $Id$


\clearpage
\section{Groupe (module \group)}

révision : \obm 2.1.0\\

\subsection{Organisation de la base de données}

Le module \group utilise 7 tables :

\begin{tabular}{|p{3cm}|p{11cm}|}
\hline
\textbf{Table} & \textbf{Description} \\
\hline
UGroup & le mot clé Group est déjà défini par MySQL \\
\hline
UserObmGroup & liaison entre les utilisateurs et les groupes \\
\hline
GroupGroup & liaison entre les groupes et les groupes \\
\hline
of\_usergroup & stockage des liens direct user - group (utile pour performance de détermination d'appartenance à des groupes hiérarchiques) \\
\hline
P\_UGroup & Automate : production \\
\hline
P\_UserObmGroup & Automate : production \\
\hline
P\_GroupGroup & Automate : production \\
\hline
\end{tabular}


\subsection{UGroup}
Table principale des informations d'un groupe. \\

\begin{tabular}{|p{3cm}|c|p{5.4cm}|p{2.6cm}|}
\hline
\textbf{Champs} & \textbf{Type} & \textbf{Description} & \textbf{Commentaire} \\
\hline
\_id & int 8 & Identifiant & Clé primaire \\
\hline
\_domain\_id & int 8 & Domaine d'appartenance & \\
\hline
\_timeupdate & timestamp 14 & Date de mise à jour & \\
\hline
\_timecreate & timestamp 14 & Date de création & \\
\hline
\_userupdate & int 8 & Id du modificateur & \\
\hline
\_usercreate & int 8 & Id du créateur & \\
\hline
\_local & int 1 & Indicateur de groupe local & A exclure des synchronisations \\
\hline
\_ext\_id & varchar 24 & Identifiant externe & Utile pour les synchronisations avec données externes\\
\hline
\_samba & int 1 & Indicateur groupe Windows & (1=Oui) \\
\hline
\_gid & int 8 & GID & identifiant du groupe (unix, windows) \\
\hline
\_mailing & int 1 & Indicateur pour liste de diffusion externe & Pour export sympa \\
\hline
\_system & int 1  & Indicateur de groupe systeme & Groupe ne peut être modifié \\
\hline
\_privacy & int 2  & Indicateur privé / public & public = 0, privé = 1\\
\hline
\_name & varchar 32 & Nom & \\
\hline
\_desc & varchar 128 & Description  & \\
\hline
\_email & varchar 128 & Adresse email du groupe & \\
\hline
\_contacts & text & Liste d'adresses email externes à ajouter & \\
\hline
\end{tabular}

\subsubsection{Groupes publics / privés : gestion des droits}

Les groupes publics sont visibles par tous les utilisateurs ayant accés au module \group.
Dans la pratique il s'agit souvent de groupes importés ou synchronisés depuis une source de données externe.

Les groupes privés sont créés par les utilisateurs ayant le droit de création de groupe (\$cright\_write) et servent principalement à définir des vues de l'agenda.

\paragraph{Dans la pratique, les droits de gestion des groupes privés doivent être dissociés des droits de gestion de groupes privés}. \\

Le modèle générique de gestion des droits d'\obm ne permet pas de dissocier le droit de créér/modifier un groupe en fonction d'un de ses attributs (privé / public).
Le droit d'écriture sur les groupes \$cright\_write donne le droit de créer/modifier un groupe qu'il soit public ou privé.

\obm implémente cette granularité supplémentaire (à partir d'\obm v. 2.0.2) :
\begin{itemize}
\item avec la règle : droit de gestion groupe public = \variable{\$cright\_write\_admin}
\item via une fonction dédiée du module groupe qui permet de tester si l'utilisateur connecté à les droits de mise à jour (créer, modifier, supprimer, ajouter ou supprimer des users ou des groupes) sur le groupe consulté.\\
\end{itemize}

Règles groupes privé / public :\\

\begin{tabular}{|p{2cm}|p{3.5cm}|p{8cm}|}
\hline
\textbf{Groupe} & \textbf{Droit pour gestion} & \textbf{Description} \\
\hline
Privé & \$cright\_write &
\begin{itemize}
\item Défini par un utilisateur,
\item Ne peut pas avoir d'adresse email 
\item Utilisé pour les vues d'agenda
\end{itemize}
\\
\hline
Public & \$cright\_write\_admin &
\begin{itemize}
\item Souvent synchronisé ou créé par import,
\item Peut devenir une adresse de diffusion (email)
\item Utilisé par l'ensemble d'\obm
\end{itemize}
\\
\hline
\end{tabular}
\vspace{0.3cm}

Fonction de test des droits de mise à jour :\\

\shadowbox{
\begin{minipage}{13cm}
\begin{verbatim}
function check_group_update_rights($params) {
\end{verbatim}
\end{minipage}
}\\

Cette fonction est appelée :
\begin{itemize}
\item par les actions : user\_add, user\_del, group\_add, group\_del
\item dans la fonction d'affichage d'un groupe pour savoir s'il faut ou non afficher la possibilité de supprimer les utilisateurs de la liste affichée
\end{itemize}
\vspace{0.3cm}

Afin de ne pas proposer les menus ``modifier, supprimer,..'' si l'utilisateur n'a pas le droit suffisant sur le groupe, la fonction \fonction{update\_group\_action()} est modifiée afin d'augmenter le droit recquis à \variable{\$cright\_write\_admin} des ces actions si le groupe consulté est public.


\subsection{UserObmGroup}
Table de liaison User - Group.\\

\begin{tabular}{|p{3cm}|c|p{5.4cm}|p{2.6cm}|}
\hline
\textbf{Champs} & \textbf{Type} & \textbf{Description} & \textbf{Commentaire} \\
\hline
\_group\_id & int 8 & Identifiant du groupe & \\
\hline
\_userobm\_id & int 8 & Identifiant de l'utilisateur & \\
\hline
\end{tabular}


\subsection{GroupGroup}
Table de liaison User - Group.\\

\begin{tabular}{|p{3cm}|c|p{5.4cm}|p{2.6cm}|}
\hline
\textbf{Champs} & \textbf{Type} & \textbf{Description} & \textbf{Commentaire} \\
\hline
\_parent\_id & int 8 & Identifiant du groupe parent & \\
\hline
\_child\_id & int 8 & Identifiant du group fils & \\
\hline
\end{tabular}


\subsection{of\_usergroup}
Table de liaison directe User - Group.\\

\begin{tabular}{|p{3cm}|c|p{5.4cm}|p{2.6cm}|}
\hline
\textbf{Champs} & \textbf{Type} & \textbf{Description} & \textbf{Commentaire} \\
\hline
\_group\_id & int 8 & Identifiant du groupe & \\
\hline
\_userobm\_id & int 8 & Identifiant de l'utilisateur & \\
\hline
\end{tabular}


\subsubsection{Remarques}

\obm gère des groupes hiérarchique (groupes de groupes) récursifs. C'est pratique en terme d'organisation et structuration de ses groupes, mais une hiérarchie de groupe peut être couteuse en performance.
\begin{itemize}
\item Déterminer l'appartenance d'un utilisateur à un groupe nécessite de parcourir toute la hiérarchie : groupes dont l'utilisateur est membre, groupes pères des groupes dont l'utilisateur est membre,... soit rapidement un grand nombre de requêtes récursives.
\item La gestion des ACL se basant sur des groupes pourra demander la vérification de plusieurs appartenances.
\end{itemize}

L'interêt de cette table est de stocker directement à plat tous les membres d'un groupe (membres directs, membres de groupes du groupe,...), afin de permettre une détermination directe et immédiate des membres d'un groupe ou des groupes dont est membre un utilisateur.

\paragraph{A l'heure actuelle, \obm 2.1.0pre3} la structure est en base de données mais n'est pas renseignée ou exploitée.
