% Documentation technique d'OBM : module Paiment
% ALIACOM Pierre Baudracco
% $Id$


\clearpage
\section{Paiment}

Le module \payment d'\obm.

\subsection{Organisation de la base de donn�es}

Le module \payment utilise 3 tables :
\begin{itemize}
 \item Payment
 \item PaymentKind
 \item PaymentInvoice
\end{itemize}

\subsection{Payment}
Table principale des informations d'un paiement.\\

\begin{tabular}{|p{3cm}|c|p{5.4cm}|p{2.6cm}|}
\hline
\textbf{Champs} & \textbf{Type} & \textbf{Description} & \textbf{Commentaire} \\
\hline
\_id & int 8 & Identifiant & Cl� primaire \\
\hline
\_timeupdate & timestamp 14 & Date de mise � jour & \\
\hline
\_timecreate & timestamp 14 & Date de cr�ation & \\
\hline
\_userupdate & int 8 & Id du modificateur & \\
\hline
\_usercreate & int 8 & Id du createur & \\
\hline
\_company\_id & int 8 & Soci�t� client ou fournisseur & \\
\hline
\_number & varchar 10 & Num�ro du paiement & \\
\hline
\_date & date & Date du paiment & \\
\hline
\_expected\_date & date & Date pr�vue du paiment & \\
\hline
\_amount & double 10,2 & Montant &\\
\hline
\_label & varchar 128 & Label & \\
\hline
\_paymentkind\_id & int 8 & Moyen de paiement & \\
\hline
\_account\_id & int 8 & Compte cr�dit� ou d�bit� & \\
\hline
\_inout & char 1 & Type entr�e ou sortie & (+ entr�e, - sortie) \\
\hline
\_checked & char 1 & Indicateur de v�rification & (1 = 0ui)\\
\hline
\_comment & text (64k) & Commentaire &\\
\hline
\end{tabular}


\subsection{PaymentKind}
Table des moyens de paiement.\\

\begin{tabular}{|p{3cm}|c|p{5.4cm}|p{2.6cm}|}
\hline
\textbf{Champs} & \textbf{Type} & \textbf{Description} & \textbf{Commentaire} \\
\hline
\_id & int 8 & Identifiant & Cl� primaire \\
\hline
\_shortlabel & char 3 & Label court & \\
\hline
\_longlabel & varchar 40 & Label & \\
\hline
\end{tabular}


\subsection{PaymentInvoice}
Table de liaison entre les paiements et les factures.\\
Une facture peut avoir plusieurs paiement et un paiement peut couvrir plusieurs factures.

\begin{tabular}{|p{3cm}|c|p{5.4cm}|p{2.6cm}|}
\hline
\textbf{Champs} & \textbf{Type} & \textbf{Description} & \textbf{Commentaire} \\
\hline
\_invoice\_id & int 8 & Identifiant de la facture & \\
\hline
\_payment\_id & int 8 & Identifiant du paiment & \\
\hline
\_amount & double 10,2 & Montant concern� du paiment pour la facture &\\
\hline
\_timeupdate & timestamp 14 & Date de mise � jour & \\
\hline
\_timecreate & timestamp 14 & Date de cr�ation & \\
\hline
\_userupdate & int 8 & Id du modificateur & \\
\hline
\_usercreate & int 8 & Id du createur & \\
\hline
\end{tabular}



\subsection{Actions et droits}

Voici la liste des actions du module \payment, avec le droit d'acc�s requis ainsi qu'une description sommaire de chacune d'entre elles.\\

\begin{tabular}{|l|c|p{9.5cm}|}
 \hline
 \textbf{Intitul�} & \textbf{Droit} & \textbf{Description} \\
 \hline
 \hline
  index & read & (D�faut) formulaire de recherche des paiments. \\ 
 \hline
  search & read & R�sultat de recherche de paiments. \\
 \hline
  new & write & Formulaire de cr�ation d'un paiment. \\
 \hline
  detailconsult & read & Fiche d�tail d'un paiment. \\
 \hline
  detailupdate & write & Formulaire de modification d'un paiment. \\
 \hline
  duplicate & write & Formulaire de cr�ation � partir d'une facture existante. \\
 \hline
  insert & write & Insertion d'un paiment. \\
 \hline
  update & write & Mise � jour d'un paiment. \\
 \hline
  check\_delete & write & V�rification avant suppression d'un paiment. \\
 \hline
  delete & write & Suppression d'un paiment. \\
 \hline
  display & read & Ecran de modification des pr�f�rences d'affichage. \\
 \hline
  dispref\_display & read & Modifie l'affichage d'un �l�ment. \\
 \hline
  dispref\_level & read & Modifie l'ordre d'affichage d'un �l�ment. \\
 \hline
\end{tabular}
