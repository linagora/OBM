% Documentation technique d'OBM : Gestion des dates
% ALIACOM Pierre Baudracco
% $Id$


\clearpage
\subsection{Gestion des dates}

Les probl�mes de format des dates sont g�n�riques et pariculi�rement sensibles dans une application comme \obm fonctionnant avec diff�rentes bases de donn�es et permettant � chaque utilisateur de s�lectionner sa propre langue.\\

Afin de simplifier la gestion des dates \obm d�finit un cadre avec des r�gles d'utilisation des dates et une api.

\subsubsection{Cadre g�n�ral et r�gles d'utilisation}

\begin{tabular}{|p{7cm}|p{7cm}|}
\hline
\textbf{R�gles} & \textbf{Implications} \\
\hline
\obm r�cup�re les dates de la base de donn�es au format \textbf{Timestamp Unix}. & La BD doit savoir retourner une date au format \textbf{Timestamp Unix}.
Voir :
\begin{itemize}
\item \fonction{sql\_date\_format()}
\end{itemize}\\
\hline
\obm propose des fonctions de formattage des dates r�cup�r�es tenant compte des param�tres utilisateur.
&
Toute date r�cup�r�e de la BD doit utiliser une des fonctions de formattage avant affichage.
Voir :
\begin{itemize}
\item \fonction{date\_format()}
\item \fonction{isodate\_format()}
\item \fonction{datetime\_format()}
\end{itemize}\\
\hline
\obm re�oit les dates � ins�rer en BD au format ISO ``\textbf{AAAA-MM-JJ HH:MM:SS}''
&
La BD doit accepter en entr�e des dates au format ISO.
A noter que la fonction javascript globale \fonction{check\_date()} permet aussi de saisir les dates au format du pays (JJ/MM/AAAA ou MM/JJ/AAAA). Cette fonction est d�finie dans le fichier de langue \fichier{check\_date.js}\\
\hline
OBM acceptant en simultan� des langues diff�rentes pour les utilisateurs, propose des labels pour le nom des mois, et jours de semaine.
&
Voir :
\begin{itemize}
\item \variable{\$l\_monthsofyear}
\item \variable{\$l\_monthsofyearshort}
\item \variable{\$l\_daysofweek}
\item \variable{\$l\_daysofweekshort}
\item \variable{\$l\_daysofweekreallyshort}
\end{itemize}\\
\hline
\end{tabular}


\subsubsection{R�cup�ration de date au format Timestamp Unix}

\paragraph{Impl�mentation multi-base de donn�es}

\fonction{sql\_date\_format()} d�finie dans le fichier \fichier{global\_query.inc} :

\shadowbox{
\begin{minipage}{13cm}
\begin{verbatim}
function sql_date_format($db_type, $field, $as="") {
  global $db_type_mysql, $db_type_pgsql;

  if ($db_type == $db_type_mysql) {
    $ret = "UNIX_TIMESTAMP($field)";
    if ($as != "") {
      $ret .= " as $as";
    }
  } elseif ($db_type == $db_type_pgsql) {
    $ret = "EXTRACT (EPOCH from $field)";
    if ($as != "") {
      $ret .= " as $as";
    }
  } else {
    $ret = $field;
  }

  return $ret;
}
\end{verbatim}
\end{minipage}
}
\paragraph{Utilisation} dans une requ�te :\\

\shadowbox{
\begin{minipage}{13cm}
\begin{verbatim}
  $obm_q = new DB_OBM;
  $db_type = $obm_q->type;
  $datealarm = sql_date_format($db_type, "deal_datealarm", "datealarm");
\end{verbatim}
\end{minipage}
}
