% Documentation technique d'OBM : Gestion des entit�s priv�es
% ALIACOM Pierre Baudracco
% $Id$


%%\clearpage
\subsection{Gestion des entit�s priv�es et visibilit�}

La gestion des entit�s priv�es est standardis�e afin de permettre une �volution possible des r�gles de visibilit� des entit�s.


\subsubsection{Sp�cifications de la notion de visibilit�}

Une api tr�s simple (1 fonction d�finie dans global.inc) est disponible :\\

\shadowbox{
\begin{minipage}{13cm}
\begin{verbatim}
is_entity_visible($entity, $obm_q, $uid='''')
\end{verbatim}
\end{minipage}
}
\begin{itemize}
 \item \textbf{\$entity} : entit� � v�rifier.\\
Exemple : ``company'', ``contact''... Attention ce n'est pas toujours le module (ex: parendeal dans module deal)
 \item \textbf{\$obm\_q} : Objet base de donn�es contenant l'entit�\\
Exemple : souvent l'objet associ� � la requete : run\_query\_detail()
 \item \textbf{\$uid} (optionnel) : uid de l'utilisateur pour lequel le droit de visibilit� doit �tre v�rifi�.
Si non donn�, l'utilisateur courant est utilis�.
\end{itemize}

\vspace{0.4cm}

Actuellement une entit� est visible par un utilisateur si l'entit� est publique (visibility == 0) ou que l'utilisateur en est le cr�ateur.

Impl�mentation du test de visibilit� :
\begin{verbatim}
  $field_vis = "${entity}_visibility";
  $field_uc = "${entity}_usercreate";

  if ( ($q->f("$field_vis") == 0)
    || ($q->f("$field_uc") == $uid) ) {
    return true;
  } else {
    return false;
  }
\end{verbatim}


\subsubsection{Modules g�rant la notion de visibilit�}

\begin{tabular}{|p{5cm}|c|}
\hline
\textbf{Module} & \textbf{Depuis version OBM} \\
\hline
contact & 0.4.0 \\
\hline
deal & 0.4.0 \\
\hline
list & 0.8.2 \\
\hline
\end{tabular}