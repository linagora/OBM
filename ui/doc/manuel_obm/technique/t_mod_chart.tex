% Documentation technique d'OBM : Module Chart
% ALIACOM Pierre Baudracco
% $Id$

\clearpage
\section{Diagrammes}

Depuis la version 1.1, \obm intègre un module de génération de diagrammes.

\obm utilise actuellement la librairie ou le projet Artichow pour la création des diagrammes.
Le module \chart propose une couche d'abstraction d'utilisation des diagrammes permettant éventuellement de changer de librairie sous-jacente.\\

Le module \chart est particulier car il n'apparait pas dans les menus de l'application mais est un service pour les modules devant générer des diagrammes.
Le module \chart est appelé via une URL, et retourne, sous forme d'image, le diagramme correspondant aux paramètres passés.

Il n'effectue pas de gestion des droits ou de session.


\subsection{Diagrammes proposés}

Artichow étant assez riche, \obm s'enrichi de nouveaux diagrammes au fur et à mesure des besoins.\\

\begin{tabular}{|p{4cm}|p{4cm}|}
\hline
\textbf{Diagramme} & \textbf{Version d'\obm} \\
\hline
Barre simple & 1.1.1 \\
\hline
Barres multiples & 1.2.0 \\
\hline
\end{tabular}


\subsection{Paramètres et utilisation}

\subsubsection{Diagramme à barre simple}

\begin{tabular}{|p{2cm}|p{3cm}|p{8cm}|}
\hline
\textbf{Paramètre} & \textbf{Valeur} & \textbf{Description}\\
\hline
action & bar & \\ 
\hline
title & texte & Titre du diagramme \\
\hline
values & tableau & Valeur des points (barres) \\
\hline
labels & tableau & Labels affichés des points (barres) \\
\hline
xlabels & tableau & Labels de l'axe X \\
\hline
\end{tabular}


\subsubsection{Diagramme à barres multiples}

\paragraph{Note :} 1 plot correspond à un ensemble de données (1 jeu de barres), 1 bar correspond à une barre. Une barre peut présenter des informations de plusieurs plots (barre découpée en parties ou couleurs, comme le potentiel des factures).\\

\begin{tabular}{|p{2cm}|p{3cm}|p{8cm}|}
\hline
\textbf{Paramètre} & \textbf{Valeur} & \textbf{Description}\\
\hline
action & bar\_multiple & \\ 
\hline
title & texte & Titre du diagramme \\
\hline
xlabels & tableau & Labels de l'axe X \\
\hline
plots & \multicolumn{2}{|c|}{tableau avec entrées du tableau suivant} \\
\hline
\end{tabular}
\vspace{0.3cm}

Entrées du hachage plots :\\

\begin{tabular}{|p{2cm}|p{3cm}|p{8cm}|}
\hline
\textbf{Paramètre} & \textbf{Valeur} & \textbf{Description}\\
\hline
values & tableau & values[0] : Valeur des points du 1er plot \\
\hline
labels & tableau & labels[0] : Labels affichés des points du 1er plot \\
\hline
legends & tableau & legend[0] : Légende associée au 1er plot \\
\hline
new\_bar & tableau (1|0) & new\_bar[n]=1 indique si plot n affiché dans une nouvelle barre (sinon affiché sur la barre actuelle) \\
\hline
\end{tabular}


\paragraph{Exemple :}

\begin{verbatim}
  $title = "$l_billed $year_prev / $year";
  $plots["new_bar"] = array(1, 1);
  $plots["legends"] = array("$year_prev", "$year");
  $plots["values"] = array($values_prev, $values);
  $plots["labels"] = array($labels_prev, $labels);

  $chart_ytoy_params .= "&amp;title=".urlencode($title).
  "&amp;plots=".urlencode(serialize($plots)) .
  "&amp;xlabels=".urlencode(serialize($xlabels));

  ...

  $block = "
    <img border=\"0\"
    src=\"$path/chart/chart_index.php?action=bar_multiple$chart_ytoy_params\"
     alt=\"\" />
  ...
\end{verbatim}
