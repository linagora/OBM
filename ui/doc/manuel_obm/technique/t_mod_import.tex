% Documentation technique d'OBM : module Import
% ALIACOM Pierre Baudracco
% $Id$


\clearpage
\section{Import}

Le module \import \obm.

\subsection{Organisation de la base de données}

Le module \import utilise 1 table :
\begin{itemize}
 \item Import
\end{itemize}

\subsubsection{La table Import}
Table principale des informations d'un import.\\

\begin{tabular}{|p{3cm}|c|p{5.4cm}|p{2.6cm}|}
\hline
\textbf{Champs} & \textbf{Type} & \textbf{Description} & \textbf{Commentaire} \\
\hline
\_id & int 8 & Identifiant & Clé primaire \\
\hline
\_timeupdate & timestamp 14 & Date de mise à jour & \\
\hline
\_timecreate & timestamp 14 & Date de création & \\
\hline
\_userupdate & int 8 & Id du modificateur & \\
\hline
\_usercreate & int 8 & Id du créateur & \\
\hline
\_name & varchar 64 & Nom de l'import & \\
\hline
\_datasource\_id & int 8 & Source de données & \\
\hline
\_marketingmanager\_id & int 8 & Responsable affecté aux données & \\
\hline
\_separator & char 3 & Séparateur de champ & \\
\hline
\_enclosed & char 1 & caractère d'encadrement des champs & \\
\hline
\_desc & text (64k) & Description de l'import (mapping des champs,...) &\\
\hline
\end{tabular}


\subsubsection{Le champ description (mapping des champs)}

L'idée globale du fonctionnement de l'import est d'effectuer un mapping des champs du fichier à importer dans les champs définis de la base de données.

Ce mappin gest stocké dans le champ \variable{import\_desc}, sous forme de définitions de variables, qui est évalué par OBM.\\

Les informations concernant les champs société sont stockées dans le tableau global \variable{\$comp}, celles concernant les champs contact dans \variable{\$con}.

Pour chaque champ défini et proposé par le module \import, le champ description stocke 3 informations : Exemple pour le nom de la société, le tableau \variable{\$comp[``comp\_name'']} contient ces 3 informations :\\

\begin{tabular}{|p{3cm}|p{5.6cm}|p{2.6cm}|}
\hline
\textbf{Champs} & \textbf{Description} & \textbf{Exemple} \\
\hline
[``value''] & numéro de colonne dans fichier & 1 \\
\hline
[``label''] & nom du label du champ & l\_company \\
\hline
[``default''] & valeur par défaut si non présent & ZZ \\
\hline
\end{tabular}


\subsection{Précisions sur le mapping des champs}

L'association des champs du fichier aux champs de la base de données utilise plusieurs structures :\\

\begin{tabular}{|p{1.5cm}|p{3.6cm}|p{4.8cm}|p{4cm}|}
\hline
\textbf{Structure} & \textbf{Description} & \textbf{Exemple} & \textbf{Fonction}\\
\hline
\$row[] & Tableau des valeurs d'une ligne du fichier & \$row[0]=Aliacom & \fonction{fgetcsv de PHP} \\
\hline
\$map[] & Mapping descriptif de l'import & \$map[l][field]=comp\_name \$map[1][label]=l\_company \$map[default][comp\_name]=ZZ \$map[field][comp\_name]=l & \fonction{get\_import\_field\_mapping}\\
\hline
\$fields[] & Valeurs des champs & \$fields[comp\_name]=Aliacom & \fonction{get\_import\_row\_mapping}\\
\hline
\end{tabular}

\vspace{0.3cm}
La structure \variable{\$row} est récupérée par une lecture de la ligne du fichier en tenant compte des paramètres CSV (séparateur, caractère encadrant).
L'utilisation de la fonction \fonction{fgetcsv()} de PHP permet de gérer les caractères encadrant optionnels (champs encadrés optionnellement par ").\\

La structure \variable{\$map} est construite à partir du mapping entré par l'utilisateur.
Elle comporte 3 sous structures :\\

\begin{tabular}{|p{2cm}|p{3.2cm}|p{4.8cm}|p{3.8cm}|}
\hline
\textbf{Structure} & \textbf{Description} & \textbf{Exemple} & \textbf{Commentaire}\\
\hline
\$map[N] & Mapping num colonne vers champ & \$map[l][field]=comp\_name \$map[1][label]=l\_company & N numéro de colonne du fichier. Utile pour afficher exemple du fichier \\
\hline
\$map[default] & Mapping valeurs des champs par défaut & \$map[default][comp\_name]=ZZ & \\
\hline
\$map[field] & Mapping champ recoit colonne & \$map[field][comp\_name]=l & Permet qu'une colonne soit mappée sur plusieurs champs\\
\hline
\end{tabular}

\vspace{0.3cm}
La structure \variable{\$fields} est construite à partir des 2 structures précédents (\variable{\$row} et \variable{\$map}. Elle contient le mapping final des valeurs dans les champs.


\subsection{Actions et droits}

Voici la liste des actions du module \import, avec le droit d'accès requis ainsi qu'une description sommaire de chacune d'entre elles.\\

\begin{tabular}{|l|c|p{9.5cm}|}
 \hline
 \textbf{Intitulé} & \textbf{Droit} & \textbf{Description} \\
 \hline
 \hline
  index & read\_admin & (Défaut) formulaire de recherche d'imports. \\ 
 \hline
  search & read\_admin & Résultat de recherche. \\
 \hline
  new & write\_admin & Formulaire de création d'un import. \\
 \hline
  detailconsult & read\_admin & Fiche détail d'un import. \\
 \hline
  detailupdate & write\_admin & Formulaire de modification d'un import. \\
 \hline
  insert & write\_admin & Insertion d'un import. \\
 \hline
  update & write\_admin & Mise à jour d'un import. \\
 \hline
  check\_delete & write\_admin & Vérification avant suppression d'un import. \\
 \hline
  delete & write\_admin & Suppression d'un import. \\
 \hline
  file\_sample & write\_admin & Vérification d'un fichier. Affichage 1eres lignes\\
 \hline
  file\_test & write\_admin & Test d'un import du fichier. \\
 \hline
  file\_import & write & Importer un fichier. \\
 \hline
\end{tabular}
