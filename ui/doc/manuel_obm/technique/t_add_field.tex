% Documentation technique d'OBM : Etendre OBM : Ajouter un champ
% ALIACOM Pierre Baudracco
% $Id$


%%\clearpage
\subsection{Ajouter un champ dans un module}

Différentes étapes sont nécessaires pour ajouter un champ dans un module. Selon l'utilisation du nouveau champ, certaines étapes peuvent êtres ignorées (ex: inclusion dans moteur de recherche, affichage dans résultat,...).\\

\begin{itemize}
 \item Modifier les scripts de création de base de données pour inclure le nouveau champ.
   \begin{verbatim}
emacs scripts/0.8/create_obmdb_0.8.mysql.sql
emacs scripts/0.8/create_obmdb_0.8.pgsql.sql
   \end{verbatim}
 \item Modifier (ou créer) les scripts de mise à jour de base de données pour inclure le nouveau champ.
   \begin{verbatim}
emacs scripts/0.8/update_0.8.1-0.8.2.mysql.sql
emacs scripts/0.8/update_0.8.1-0.8.2.pgsql.sql
   \end{verbatim}
 \item (module\_index.php) Les paramètres sont récoltés automatiquement. Mais si le nouveau paramètre doit subir un traitement spécial, ceci doit être implémenté dans (\fonction{get\_module\_params()})\\
 \item Si champ sélection d'une liste déroulante :
    \begin{itemize}
     \item[*] Créer la table associée (scripts création et mise à jour bases de données)\\
     \item[*] (module\_index.php) Appeler la liste de valeurs run\_query\_fieldname() pour la passer aux fonctions formulaire (mise à jour voire recherche).\\
     \item[*] (module\_display.inc) Modifier la définition des fonctions formulaire afin d'inclure la liste de valeur (html\_module\_form(), html\_module\_search\_form()).\\
     \item[*] (module\_query.inc) Implémenter les fonctions run\_query\_fieldname(), run\_query\_fieldname\_insert(), run\_query\_fieldname\_update(), run\_query\_fieldname\_delete(), run\_query\_fieldname\_links().
      \begin{verbatim}
function run_query_companynafcode()
function run_query_nafcode_insert()
function run_query_nafcode_update()
function run_query_nafcode_delete()
function run_query_nafcode_links()
      \end{verbatim}
     \item[*] (module\_display.inc) Implémenter les fonctions de gestion et visualisation : html\_module\_field\_form(), dis\_field\_links().
      \begin{verbatim}
function html_company_nafcode_form()
function dis_nafcode_links()
      \end{verbatim}
     \item[*] (module\_index.php) Définir les actions avec les droits associés (get\_module\_action()) et les implémenter dans le branchement global du module (if \$action ==...) avec la traduction des messages (insert\_ok, insert\_error,...).\\
     \item[*] (module\_js.inc) Implémenter les actions Javascript de vérification des formulaire de gestion (check\_field\_upd(), check\_field\_new(), check\_field\_checkdel()).\\
    \end{itemize}
 \item (lang/*/module.inc) Créer les variables de langues (au moins pour le français et l'anglais).\\
 \item Si le champ peut être masqué, indiquer l'option dans obminclude/global.inc et obm\_conf.inc et mettre la doc à jour (manuel\_obm/technique/t\_site.tex).\\
 \item (module\_display.inc) Prendre en compte le nouveau champ dans les fonctions d'affichage (html\_module\_form(), html\_module\_consult(), dis\_module\_warn\_insert()).\\
 \item Si le champ peut être recherché l'implémenter dans les fonctions html\_module\_search\_form(), html\_module\_search\_list() pour propagation du résultat, et dans la fonction de recherche run\_query\_search().\\
 \item (module\_query.inc) Prendre en compte le nouveau champ dans les fonctions base de données : run\_query\_detail(), run\_query\_insert(), run\_query\_update(), dis\_module\_warn\_insert()).\\
 \item Si le champ doit figuré dans les résultats de recherche, il faut l'ajouter au module correspondant dans les préférences par défaut (scripts/0.8/obmdb\_default\_values\_0.8.sql).
\end{itemize}
