% Manuel d'utilisation d'OBM : module Todo
% ALIACOM Pierre Baudracco
% $Id$

\clearpage
\section{Le module \todo}

\subsection{Présentation du module \todo}

Ce module est une gestion simple de tâches de type "Post-it" permettant de définir des tâches en les affectant à une personne.\\

Les tâches en cours de l'utilisateur sont affichées dans tous les écrans d'OBM, triées par priorité ou ordre chronologique d'insertion.

\subsection{Description d'une tâche}

Composantes d'une tâche :\\

\begin{tabular}{|p{3cm}|p{10cm}|}
\hline
\textbf{Nom} & \textbf{Description} \\
\hline
Titre & Titre de la tâche.\\
\hline
Utilisateur & Utilisateur à qui la tâche est affectée.\\
\hline
Priorité & Priorité de la tâche. Les tâches peuvent être triées par priorité.\\
\hline
Date limite & Echéance de la tâche.\\
\hline
Description & Description de la tâche.\\
\hline
\end{tabular}
\vspace{0.3cm}


\subsection{Les sous-menus du module \todo}

Le module \todo comporte 2 sous-menus (accessibles selon les droits d'accès) :\\

\begin{tabular}{|p{2.5cm}|p{9.5cm}|}
\hline
\textbf{Nom} & \textbf{Action / Description} \\
\hline
Liste & Affiche la liste des tâches ainsi que le formulaire d'insertion de nouvelle  tâche.\\
\hline
Affichage & Personnalisation de l'affichage de la liste des tâches.\\
\hline
\end{tabular}


\subsubsection{Le sous-menu : Liste}

Ce menu est l'écran principal et quasi unique du module tâche.
Il permet de visionner la liste des tâches affectées à l'utilisateur.
Il affiche aussi le formulaire d'insertion de nouvelle tâche.\\

\paragraph{Création d'une nouvelle tâche}

Pour créer une nouvelle tâche il est nécessaire de donner un titre à la tâche.\\

Une tâche peut être affectée à un ou plusieurs utilisateurs.
Une tâche affectée à plusieurs utilisateurs est gérée comme autant de tâches distinctes pour chaque utilisateur. Il n'y a pas de liaison entre ces mêmes tâches.
Chaque utilisateur gère donc de façon autonome cette nouvelle tâche.


\subsubsection{Le sous-menu : Affichage}

Ce menu permet de paramétrer l'affichage de la liste des tâches. Il est possible de choisir les champs à afficher et de régler leur ordre d'affichage.
